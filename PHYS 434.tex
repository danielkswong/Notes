\documentclass[12pt,a4paper,titlepage]{article}
% Daniel Wong September 10, 2020

\usepackage[fleqn]{amsmath} % To write aligned equations
\usepackage{amssymb} % To write math symbols
\usepackage{graphicx,float} % To add pictures
\usepackage{caption} % To add captions
\usepackage{subcaption} % To add subcaptions
\usepackage{multirow} % To add multirow lines in tables
\usepackage{multicol} % To add multicolumn lines in tables
\usepackage{hyperref} % To add hyperlinks
\usepackage[normalem]{ulem} % To underline
\usepackage[backend=bibtex,citestyle=numeric,autocite=plain,sorting=none]{biblatex} % To add references
\usepackage[english]{babel} % To write accented vowels
\usepackage[toc,page]{appendix} % To add table of contents
\usepackage{physics} % To use supported physics notation
\usepackage[none]{hyphenat} % To remove hyphenation
\usepackage{mathtools}  % To use \mathclap which removes white space for subscripts
\usepackage{tensor} % For subscripts to the left

\usepackage{fancyhdr} % To input custom headers
\pagestyle{fancy} % Defines header for all pages other than plain-style pages
\fancyhf{}
\rhead{\nouppercase\leftmark}
\lfoot{Daniel Wong}
\rfoot{\thepage}
\renewcommand{\headrulewidth}{0.4pt} % Defining width of header line
\renewcommand{\footrulewidth}{0.4pt} % Defining width of footer line

\usepackage{tikz} % To input mathematical plots and diagrams
\usetikzlibrary{decorations.pathreplacing,decorations.pathmorphing,decorations.markings} % For curly braces, wavy lines, and arrow markings
\usetikzlibrary{tikzmark} % For adding annotations to equations
\usetikzlibrary{shapes.geometric} % For drawing regular polygons
\usetikzlibrary{shapes.misc} % For drawing crosses
\usetikzlibrary{snakes} % For drawing zigzag lines

\tikzset{->-/.style={decoration={markings,mark=at position #1 with {\arrow{>}}},postaction={decorate}}} 
\tikzset{-<-/.style={decoration={markings,mark=at position #1 with {\arrow{<}}},postaction={decorate}}} % Defining lines with arrows in the middle
\tikzset{cross/.style={cross out,draw,minimum size=2*(#1-\pgflinewidth),inner sep=0pt,outer sep=0pt}} % Defining cross for a node

\usepackage{color} % To use different colours
\usepackage[makeroom]{cancel} % To cancel terms in equations

\definecolor{lightGray}{gray}{0.8}

\newcommand{\trm}[1]{\textrm{#1}} % Shorthand for \textrm command
\newcommand{\up}{\uparrow} % Shorthand for \uparrow command
\newcommand{\dn}{\downarrow} % Shorthand for \downarrow command
\newcommand{\en}{\epsilon_{0}} % Shorthand for \epsilon_{0}
\newcommand{\ul}[1]{\underline{\smash{#1}}} % Properly underlines words
\newcommand{\explain}[1]{% Adds an arrow to allow for explanations in equations
	\quad\rotatebox[origin=c]{180}{\enskip$\Lsh$}\hfill
	\begin{minipage}[t]{\dimexpr\linewidth-8em\relax}
	#1
	\end{minipage}\hspace{4em}\bigskip
}

\newcommand{\aside}[1]{% Aligns the environment for asides
	\ul{Aside}:\hfill
	\begin{minipage}[t]{\dimexpr\linewidth-8em\relax}
	#1
	\end{minipage}\hspace{4em}\bigskip
}
\newcommand{\definition}[1]{% Aligns the environment for definitions
	\ul{Definition}:\hfill
	\begin{minipage}[t]{\dimexpr\linewidth-8em\relax}
	#1
	\end{minipage}\hspace{4em}\bigskip
}
\newcommand{\example}[2]{% Aligns the environment for examples
	\bigskip
	\hrule
	\bigskip
	\ul{Ex #1}:\hfill
	\begin{minipage}[t]{\dimexpr\linewidth-8em\relax}
	#2
	\end{minipage}\hspace{4em}
	\bigskip
	\hrule
	\bigskip
}

\newcommand{\angstrom}{\textup{\AA}} % Angstrom symbol
\newcommand{\Chi}{\mathcal{X}} % Inline chi symbol
\newcommand{\sign}{\trm{sign}}
\renewcommand{\Re}{\trm{Re}} % Real
\renewcommand{\Im}{\trm{Im}} % Imaginary
\renewcommand{\CancelColor}{\color{red}} % Changes the cancel colour to red

\begin{document}
\title{PHYS 434: Quantum Physics III}
\author{Daniel Wong\\University of Waterloo\\Instructor: Dr. Anton Burkov}
\date{Fall 2017}
\maketitle

\setlength\parindent{0pt}
\pagenumbering{roman}
\numberwithin{equation}{section}

\section*{Disclaimer}
These notes may be freely used by anyone who comes across them. If any errors are found within these notes, please email me at \ul{temp@temp.com}. If you are the previous instructor of this course (Dr. Anton Burkov) and have concerns about keeping these notes on my website, please contact me at the email provided.\\

Other course notes may be found on my website at \url{www.temp.com/notes}.

\newpage
\tableofcontents
\newpage
\pagenumbering{arabic}

\section{Review of Quantum Mechanics}
\subsection{Discrete Spectrum}
States of the system in quantum mechanics are vectors in Hilbert space $\mathcal{H}$. The Hilbert space $\mathcal{H}$ is the space of all states of a given system. In Dirac notation, the states are called kets $\ket{\psi}$.\\

Observables in quantum mechanics are operators $\vb{A}:\mathcal{H}\rightarrow\mathcal{H}$ acting on kets such that $\vb{A}\ket{\psi}=\ket{\varphi}$. Every operator $\vb{A}$ has a set of eigenstates (or eigenkets) $\{\ket*{a'}\}$ where
\begin{equation}
\begin{aligned}
\vb{A}\ket*{a'}&=a'\ket*{a'}\\
\vb{A}\ket*{a''}&=a''\ket*{a''}
\end{aligned}
\end{equation}
The eigenvalue corresponding to the eigenket $\ket*{a'}$ is denoted $a'\in\mathbb{R}$.\\

The dual Hilbert space to the ket space is the bra space. Elements of the bra space are bras and are denoted as $\bra{\varphi}$.\\

One can now define an inner product (or scalar product) of a bra $\bra{\varphi}$ and ket $\ket{\psi}$ to be $\braket{\varphi}{\psi}$. By definition
\begin{equation}
\braket{\varphi}{\psi}=\braket{\psi}{\varphi}^{*}
\end{equation}
This implies that $\braket{\psi}{\psi}=\braket{\psi}{\psi}^{*}$ is real. We postulate that $\braket{\psi}{\psi}$ is non-negative, meaning
\begin{equation}
\norm{\psi}=\braket{\psi}{\psi}^{\frac{1}{2}}\geq0
\end{equation}

Thus, every state in the Hilbert space can be normalized.
\begin{equation}
\begin{aligned}
\ket*{\tilde{\psi}}&=\frac{1}{\sqrt{\braket{\psi}{\psi}}}\ket{\psi}\\
\implies \braket*{\tilde{\psi}}{\tilde{\psi}}&=\frac{\braket{\psi}{\psi}}{\braket{\psi}{\psi}}=1
\end{aligned}
\end{equation}
If one has that $\braket{\varphi}{\psi}=0$, then $\ket{\psi}$ and $\ket{\varphi}$ are orthogonal.\\

The dual of $\vb{A}\ket{\psi}$ is $\bra{\psi}\vb{A}^{\dagger}$, where $\vb{A}^{\dagger}$ is the Hermitian adjoint of $\vb{A}$. One can act on the ket $\vb{A}\ket{\psi}$ with the bra $\bra{\varphi}$ and obtain
\begin{equation}
\mel{\varphi}{\vb{A}}{\psi}=\mel{\psi}{\vb{A}^{\dagger}}{\varphi}^{*}
\end{equation}

The operator is Hermitian if $\vb{A}^{\dagger}=\vb{A}$. For $\vb{A}$ as a Hermitian operator, let $(a',\ket*{a'})$ and $(a'',\ket*{a''})$ be two eigenpairs. Thus
\begin{equation}
\begin{aligned}
\vb{A}\ket*{a'}&=a'\ket*{a'}\\
\vb{A}\ket*{a''}&=a''\ket*{a''}
\end{aligned}
\end{equation}

Let $\bra{\varphi}$ be an arbitrary bra. Thus
\begin{equation}
\begin{aligned}
\mel*{\varphi}{\vb{A}}{a''}&=\mel*{\varphi}{a''}{a''}=a''\braket*{\varphi}{a''}\\
\mel*{a''}{\vb{A}}{\varphi}&=\mel*{\varphi}{\vb{A}}{a''}^{*}=a''^{*}\braket*{a''}{\varphi}
\end{aligned}
\end{equation}
Since $\bra{\varphi}$ is arbitrary
\begin{equation}
\bra*{a''}\vb{A}=a''^{*}\bra*{a''}
\end{equation}
From $\vb{A}\ket*{a'}=a'\ket*{a'}$ and $\bra*{a''}\vb{A}=a''^{*}\bra*{a''}$
\begin{equation}
\begin{aligned}
\mel*{a''}{\vb{A}}{a'}=a'\braket*{a''}{a'}=a''^{*}\braket*{a''}{a'}\\
\implies (a'-a''^{*})\braket*{a''}{a'}=0
\end{aligned}
\end{equation}
One can choose $\ket*{a''}=\ket*{a'}$ to show
\begin{equation}
(a'-a'^{*})\braket*{a'}{a'}=0 \implies a'=a'^{*}
\end{equation}
since $\braket*{a'}{a'}>0$ with $\ket*{a'}\neq\ket{0}$. Thus, all Hermitian operators have real eigenvalues. Since the spectrum of an operator correspond to physical observales, this is in agreement with the fact that all physical quantities are real-valued.\\

Consider now $\ket*{a'}$ and $\ket*{a''}$ to be different and non-degenerate (different eigenvalues). Thus
\begin{equation}
\begin{aligned}
(a'-a''^{*})\braket*{a''}{a'}&=0\\
(a'-a'')\braket*{a''}{a'}&=0\trm{, since all eigenvalues are real}\\
\braket*{a''}{a'}&=0\trm{, since $\ket*{a'}$ and $\ket*{a''}$ are non-degenerate}
\end{aligned}
\end{equation}
Thus, eigenstates corresponding to different eigenvalues are orthogonal. We shall also assert all eigenstates are normalized, since their norm is arbitrary. Thus, these results can be represented as 
\begin{equation}
\braket*{a'}{a''}=\delta_{a'a''}
\end{equation}

Thus, the set of eigenkets of any Hermitian operator forms a complete orthonormal set of states and act as a basis for the Hilbert space. Thus, any ket $\ket{\psi}$ can be written as a linear combination of eigenkets for any Hermitian operator $\vb{A}$.
\begin{equation}
\ket{\psi}=\sum_{a'}c_{a'}\ket{a'}
\end{equation}
Using the orthonormality of eigenkets
\begin{equation}
\begin{aligned}
\braket*{a''}{\psi}&=\sum_{a'}c_{a'}\braket*{a''}{a'}=\sum_{a'}c_{a'}\delta_{a''a'}=c_{a''}\\
\implies c_{a'}&=\braket*{a'}{\psi}\in\mathbb{C}
\end{aligned}
\end{equation}
Physically, $c_{a'}$ is the probability amplitude. For a sysem in state $\ket{\psi}$, the probability of measuring the value $a'$ when making measurement $\vb{A}$ is the square modulus of $c_{a'}$ (i.e. $\abs{c_{a'}}^{2}$). Equivalently, this is the same probability of finding state $\ket{\psi}$ in state $\ket*{a'}$ after measurement.\\

Writing $\ket{\psi}$ as a spectral decomposition gives
\begin{equation}
\ket{\psi}=\sum_{a'}\ket*{a'}\ip*{a'}{\psi}\trm{, since $c_{a'}=\ip*{a'}{\psi}$}
\end{equation}

Since $\ket{\psi}$ is arbitrary, one now obtains the closure relation (or resolution of identity).
\begin{equation}
\sum_{a'}\op*{a'}{a'}=\mathbb{I}
\end{equation}

An individual term in this sum is called the projection operator.
\begin{equation}
\begin{aligned}
\Lambda_{a'}&=\op*{a'}{a'}\\
\Lambda_{a'}\ket{\psi}&=\ket*{a'}\ip*{a'}{\psi}&=c_{a'}\ket{a'}
\end{aligned}
\end{equation}
As such, $\Lambda_{a'}$ projects $\ket{\psi}$ into the direction of eigenket $\ket{a'}$.\\

Using the closure relation, one can obtain the spectral decomposition of operator $\vb{A}$.
\begin{equation}
\vb{A}\op*{a'}{a'}=a'\op*{a'}{a'}
\end{equation}
Taking the sum over all eigenkets
\begin{equation}
\vb{A}\sum_{a'}\op*{a'}{a'}=\vb{A}\mathbb{I}=\vb{A}=\sum_{a'}\op*{a'}{a'}
\end{equation}

Consider another operator $\vb{B}$.
\begin{equation}
\vb{B}=\mathbb{I}\vb{B}\mathbb{I}=\sum_{a'a''}\op*{a''}{a''}\vb{B}\op*{a'}{a'}
\end{equation}
$\mel*{a''}{\vb{B}}{a'}$ can be interpreted as a matrix indexed by $\ket*{a''}$ and $\ket*{a'}$.

\begin{equation}
\vb{B}=\mqty(\mel*{a^{(1)}}{\vb{B}}{a^{(1)}} & \ldots & \mel*{a^{(1)}}{\vb{B}}{a^{(n)}} \\ \vdots & \ddots & \vdots \\ \mel*{a^{(n)}}{\vb{B}}{a^{(1)}} & \ldots & \mel*{a^{(n)}}{\vb{B}}{a^{(n)}})
\end{equation}
The matrix that corresponds to $\vb{B}^{\dagger}$ is the complex conjugate transposed of the matrix corresponding to $\vb{B}$.

\subsection{Continuous Spectrum}
Let us now generalize to operators with continuous spectrum. For example, the position and momentum operators are two such operators. Let $\ket*{x'}$ be a position eigenket corresponding to the state of a particle at position $x'$ in space. Let $\vb{x}$ be the position operator defined as
\begin{equation}
\vb{x}\ket*{x'}=x'\ket*{x'}
\end{equation}
Define $\psi(x')$ as the probability amplitude to find a particle in a state $\ket{\psi}$ at position $x'$. It is given by
\begin{equation}
\psi(x')=\ip*{x'}{\psi}
\end{equation}
This is equal to the wavefunction.\\

For the continuous spectra, instead of using sums, one uses integrals. Generalizing the Kronecker delta equation to the continuous spectra, one obtains the Dirac delta function $\delta(x)$
\begin{equation}
\ip*{x'}{x''}=\delta(x'-x'')
\end{equation}
where the Dirac delta is defined as
\begin{equation}
\int_{-\infty}^{\infty}\dd{x'}f(x')\delta(x')=f(0)
\end{equation}

\begin{tikzpicture}
	\draw[<->] (-2,0) -- (2,0) node[right] {$x$};
	\draw[->] (0,0) -- (0,2) node[above] {$y$};
	\draw[red,thick,fill opacity=0.2,text opacity=1] (-2,0) -- (0,0) -- (0,2) -- (0,0) -- (2,0);
	\node at (0.5,1.5) {$\delta(x)$};
\end{tikzpicture}

The closure relation then becomes
\begin{equation}
\mathbb{I}=\int_{-\infty}^{\infty}\dd{x'}\op*{x'}{x'}
\end{equation}
Thus
\begin{equation}
\ket{\psi}=\mathbb{I}\ket{\psi}=\int\dd{x'}\ket*{x'}\ip*{x'}{\psi}
\end{equation}

Let $\ket{\varphi}$ be another ket in the same Hilbert space as $\ket{\psi}$.
\begin{equation}
\begin{aligned}
\ip{\varphi}{\psi}&=\int_{-\infty}^{\infty}\dd{x'}\ip*{\varphi}{x'}\ip*{x'}{\psi}\\
&=\int_{-\infty}^{\infty}\dd{x'}\ip*{x'}{\varphi}^{*}\ip*{x'}{\psi}\\
&=\int_{-\infty}^{\infty}\dd{x'}\varphi(x')^{*}\psi(x')
\end{aligned}
\end{equation}
Compare this to $\ip{\varphi}{\psi}=\sum_{a'}\ip*{\varphi}{a'}\ip*{a'}{\psi}$ in the discrete case.

\subsection{Infinitesimal Translations}
Introduce the infinitesimal translation operator $\vb{T}$ as
\begin{equation}
\vb{T}(\dd{x'})\ket*{\vec{x}\,'}=\ket*{\vec{x}\,'+\dd{\vec{x}\,'}}
\end{equation}
where $\dd{\vec{x}\,'}$ is an infinitesimally small vector. Acting on an arbitrary $\ket{\psi}$ state
\begin{equation}
\begin{aligned}
\vb{T}(\dd{\vec{x}\,'})\ket{\psi}&=\vb{T}(\dd{\vec{x}\,'})\underbrace{\int_{\mathbb{R}}\dd[3]{x'}\op*{\vec{x}\,'}{\vec{x}\,'}}_{\mathbb{I}}\ket*{\psi}\qq{(Note that $\dd{\vec{x}\,'}\neq\dd[3]x'$)}\\
&=\int_{\mathbb{R}}\dd[3]{x'}\ket*{\vec{x}\,'+\dd{\vec{x}\,'}}\ip*{\vec{x}\,'}{\psi}\\
&=\int_{\mathbb{R}}\dd[3]{x''}\ket*{\vec{x}\,''}\ip*{\vec{x}\,''-\dd{\vec{x}\,'}}{\psi},\qq{$\vec{x}\,''=\vec{x}\,'-\dd{\vec{x}\,'}$}
\end{aligned}
\end{equation}

Without loss of generality, let $\ket{\psi}$ be normalized (i.e. $\ip{\psi}{\psi}=1$). It is useful to define $\vb{T}(\dd{\vec{x}\,'})$ such that $\vb{T}(\dd{\vec{x}\,'})\ket{\psi}$ is also normalized. Thus
\begin{equation}
\mel*{\psi}{\vb{T}^{\dagger}(\dd{\vec{x}\,'})\vb{T}(\dd{\vec{x}\,'})}{\psi}=1
\end{equation}
If this equation is to hold for arbitrary $\ket{\psi}$, then $\vb{T}(\dd{\vec{x}\,'})$ must be a unitary operator.
\begin{equation}
\begin{aligned}
\vb{T}^{\dagger}(\dd{\vec{x}\,'})\vb{T}(\dd{\vec{x}\,'})&=\mathbb{I}\\
\implies\vb{T}^{\dagger}(\dd{\vec{x}\,'})&=\vb{T}^{-1}(\dd{\vec{x}\,'})
\end{aligned}
\end{equation}

Likewise, translations are additive.
\begin{equation}
\vb{T}(\dd{\vec{x}\,''})\vb{T}(\dd{\vec{x}\,'})=\vb{T}(\dd{\vec{x}\,'}+\dd{\vec{x}\,''})
\end{equation}
Letting $\dd{\vec{x}\,''}=-\dd{\vec{x}\,'}$
\begin{equation}
\begin{aligned}
\vb{T}(-\dd{\vec{x}\,'})\vb{T}(\dd{\vec{x}\,'})&=T(\vec{0})=\mathbb{I}\\
\vb{T}(-\dd{\vec{x}\,'})&=\vb{T}^{-1}(\dd{\vec{x}\,'})
\end{aligned}
\end{equation}

All of these properties are satisfied if
\begin{equation}
\vb{T}(\dd{\vec{x}\,'})=\mathbb{I}-i\vec{\vb{K}}\cdot\dd{\vec{x}\,'},\qq{where $\vec{\vb{K}}=(\vb{K}_{x},\vb{K}_{y},\vb{K}_{z})$ is a Hermitian operator}
\end{equation}

Thus
\begin{equation}
\begin{aligned}
\vb{T}^{\dagger}(\dd{\vec{x}\,'})\vb{T}(\dd{\vec{x}\,'})&=(\mathbb{I}+i\vec{\vb{K}}^{\dagger}\cdot\dd{\vec{x}\,'})(\mathbb{I}-i\vec{\vb{K}}\cdot\dd{\vec{x}\,'})\\
&=(\mathbb{I}+i\vec{\vb{K}}\cdot\dd{\vec{x}\,'})(\mathbb{I}-i\vec{\vb{K}}\cdot\dd{\vec{x}\,'})\\
&=\mathbb{I}+(\vec{\vb{K}}\cdot\dd{\vec{x}\,'})^{2}\\
&=\mathbb{I}+\cancelto{0}{\order{\dd{\vec{x}\,'}^{2}}}\qq{where higher-order infinitesimals can be neglected}\\
&\approx\mathbb{I}
\end{aligned}
\end{equation}

Similarly
\begin{equation}
\begin{aligned}
\vb{T}(\dd{\vec{x}\,''})\vb{T}(\dd{\vec{x}\,'})&=(\mathbb{I}-i\vec{\vb{K}}\cdot\dd{\vec{x}\,''})(\mathbb{I}-i\vec{\vb{K}}\cdot\dd{\vec{x}\,'})\\
&=\mathbb{I}-i\vec{\vb{K}}\cdot\dd{\vec{x}\,''}-i\vec{\vb{K}}\cdot\dd{\vec{x}\,'}+\cancelto{0}{\order{\dd{\vec{x}\,'}^{2}}}\\
&\approx\mathbb{I}-i\vec{\vb{K}}\cdot(\dd{\vec{x}\,''}+\dd{\vec{x}\,'})\\
&=\vb{T}(\dd{\vec{x}\,''}+\dd{\vec{x}\,'})
\end{aligned}
\end{equation}

Thus, this representation satisfies both the unitary and additive properties.\\

To demonstrate the specific form of $\vec{\vb{K}}$, calculate the commutator $\comm{\vec{\vb{x}}}{\vb{T}(\dd{\vec{x}\,'})}$ between operators $\vec{\vb{x}}$ and $\vb{T}(\dd{\vec{x}\,'})$.
\begin{equation}
\begin{aligned}
\comm{\vec{\vb{x}}}{\vb{T}(\dd{\vec{x}\,'})}\ket*{\vec{x}\,'}&=\vec{\vb{x}}\vb{T}(\dd{\vec{x}\,'})\ket*{\vec{x}\,'}-\vb{T}(\dd{\vec{x}\,'})\vec{x}\,'\ket*{\vec{x}\,'}\\
&=\vec{\vb{x}}\ket*{\vec{x}\,'+\dd{\vec{x}\,'}}-\vb{T}(\dd{\vec{x}\,'})\vec{x}\,'\ket*{\vec{x}\,'}\\
&=(\vec{x}\,'+\dd{\vec{x}\,'})\ket*{\vec{x}\,'+\dd{\vec{x}\,'}}-\vec{x}\,'\vb{T}(\dd{\vec{x}\,'})\ket*{\vec{x}\,'}\\
&=(\vec{x}\,'+\dd{\vec{x}\,'})\ket*{\vec{x}\,'+\dd{\vec{x}\,'}}-\vec{x}\,'\ket*{\vec{x}\,'+\dd{\vec{x}\,'}}\\
&=\dd\vec{x}\,'\ket*{\vec{x}\,'+\dd{\vec{x}\,'}}\\
&\approx\dd\vec{x}\,'\ket*{\vec{x}\,'}\qq{given that $\dd\vec{x}\,'\ket*{\dd{\vec{x}\,'}}$ is $\order{\dd{\vec{x}\,'}^{2}}$}
\end{aligned}
\end{equation}

Therefore
\begin{equation}
\comm{\vec{\vb{x}}}{\vb{T}(\dd{\vec{x}\,'})}=\dd{\vec{x}\,'}\implies -i\vec{\vb{x}}\vec{\vb{K}}\cdot\dd{\vec{x}\,'}+i\vec{\vb{K}}\cdot\dd{\vec{x}\,'}\vec{\vb{x}}=\dd{\vec{x}\,'}
\end{equation}

Choosing $\dd{\vec{x}\,'}=\dd{x'}\hat{x}_{j}$ where $\hat{x}_{j}$ is the unit vector in the $j$-th direction
\begin{equation}
\begin{aligned}
\comm{\vec{\vb{x}}}{\vb{T}(\dd{\vec{x}\,'})}&=-i\vec{\vb{x}}\vb{K}_{j}\dd{x'}+i\vb{K}_{j}\dd{x'}\vec{\vb{x}}\\
\implies\comm{\vec{\vb{x}}}{\vb{T}(\dd{\vec{x}\,'})}_{i}&=-i\vb{x}_{i}\vb{K}_{j}\dd{x'}+i\vb{K}_{j}\dd{x'}\vb{x}_{i}=\delta_{ij}\dd{x'}\\
&\quad\explain{Since $\comm{\vec{\vb{x}}}{\vb{T}(\dd{\vec{x}\,'})}=\dd{\vec{x}\,'}=\dd{x'}\hat{x}_{j}$}\\
\implies -i\comm{\vb{x}_{i}}{\vb{K}_{j}}&=\delta_{ij}\\
\comm{\vb{x}_{i}}{\vb{K}_{j}}&=i\delta_{ij}\\
\end{aligned}
\end{equation}

Thus, $\vec{\vb{K}}=\frac{1}{\hbar}\vec{\vb{p}}$, where $\vec{\vb{p}}$ is the generator of infinitesimal translations (or the momentum operator). This gives
\begin{equation}
\comm{\vb{x}_{i}}{vb{p}_{j}}=i\hbar\delta_{ij}
\end{equation}
and
\begin{equation}
\vb{T}(\dd{\vec{x}\,'})=\mathbb{I}-\frac{i}{\hbar}\vec{\vb{p}}\cdot\dd{\vec{x}\,'}
\end{equation}

Suppose $\vec{a}$ is finite. One finds the form of the translation operator $\vb{T}(\dd{\vec{x}\,'})$ as
\begin{equation}
\begin{aligned}
\vb{T}(\vec{a})&=\lim_{N\rightarrow\infty}\vb{T}\qty(\frac{\vec{a}}{N})^{N}\\
\implies\vb{T}(\dd{\vec{x}\,'})&=\lim_{N\rightarrow\infty}\qty(\mathbb{I}-\frac{i}{\hbar}\vec{\vb{p}}\cdot\frac{\vec{a}}{N})^{N}=e^{-\frac{i\vec{\vb{p}}\cdot\vec{a}}{\hbar}}
\end{aligned}
\end{equation}

\subsection{Position and Momentum Representation Transformations}
Let us consider a 1D system, where $\Delta x'$ is infinitesimally small
\begin{equation}
\begin{aligned}
\vb{T}(\Delta x')\ket{\psi}&=\qty(1-\frac{i}{\hbar}\vb{p}\Delta x')\ket{\psi}\\
&=\int_{-\infty}^{\infty}\dd{x'}\qty(1-\frac{i}{\hbar}\vb{p}\Delta x')\ket{x'}\ip{x'}{\psi}\\
&=\int_{-\infty}^{\infty}\dd{x'}\vb{T}(\Delta x')\ket{x'}\ip{x'}{\psi}\\
&=\int_{-\infty}^{\infty}\dd{x'}\ket{x'+\Delta x'}\ip{x'}{\psi}\\
&=\int_{-\infty}^{\infty}\dd{x'}\ket{x'}\ip{x'-\Delta x'}{\psi}\\
&\quad\explain{Redefining $x'+\Delta x'$ as $x'$}
\end{aligned}
\end{equation}
Note that
\begin{equation}
\begin{aligned}
\ip{x'-\Delta x'}{\psi}&=\psi(x'-\Delta x')\\
&=\psi(x')-\Delta x'\pdv{x'}\psi(x')+\ldots\qq{by Taylor expansion}
\end{aligned}
\end{equation}
Thus
\begin{equation}
\begin{aligned}
\vb{T}(\Delta x')\ket{\psi}&=\int_{-\infty}^{\infty}\dd{x'}\ket{x'}\qty(\ip{x'}{\psi}-\Delta x'\pdv{x'}\ip{x'}{\psi})\\
&=\qty(1-\Delta x'\pdv{x'})\ket{\psi}
\end{aligned}
\end{equation}
Compare this with $\vb{T}(\Delta x')\ket{\psi}=\qty(1-\frac{i}{\hbar}\vb{p}\Delta x')\ket{\psi}$
\begin{equation}
\vb{p}\ket{\psi}=\int\dd{x'}\ket{x'}\qty(-i\hbar\pdv{x'}\ip{x'}{\psi})
\end{equation}
Equivalently, in 3D
\begin{equation}
\vec{\vb{p}}\ket{\psi}=-i\hbar\vec{\nabla}\ket{\psi}=-i\hbar\qty(\pdv{x}\hat{x}+\pdv{y}\hat{y}+\pdv{z}\hat{z})\ket{\psi}
\end{equation}
Therefore
\begin{equation}
\begin{aligned}
\mel{x''}{\vb{p}}{\psi}&=\int\dd{x'}\ip{x''}{x'}\qty(-i\hbar\pdv{x'}\ip{x'}{\psi})\\
&=\int\dd{x'}\delta(x''-x')\qty(-i\hbar\pdv{x'}\ip{x'}{\psi})\\
&=-i\hbar\pdv{x''}\ip{x''}{\psi}\\
\therefore \mel{x'}{\vb{p}}{\psi}&=-i\hbar\pdv{x'}\ip{x'}{\psi}
\end{aligned}
\end{equation}
Since a given ket $\ket{\psi}$ can be written in any basis or representation, let us transform $\ket{\psi}$ in the momentum basis. Recall the momentum eigenkets form a complete orthonormal set of states
\begin{equation}
\vb{p}\ket{p'}=p'\ket{p'}\qq{with}\ip{p'}{p''}=\delta(p'-p'')
\end{equation}
Moreover, by resolution of identity
\begin{equation}
\mathbb{I}=\int\dd{p'}\op{p'}
\end{equation}
Therefore
\begin{equation}
\begin{aligned}
\ket{\psi}&=\mathbb{I}\ket{\psi}\\
&=\int\dd{p'}\ket{p'}\ip{p'}{\psi}
\end{aligned}
\end{equation}
Thus, we define the wavefunction in momentum representation as
\begin{equation}
\ip{p'}{\psi}=\psi(p')
\end{equation}
To transform from $\psi(x')$ to $\psi(p')$, consider $\ket{\psi}=\ket{p'}$.
\begin{equation}
\begin{aligned}
\mel{x'}{\vb{p}}{\psi}&=-i\hbar\pdv{x'}\ip{x'}{\psi}\\
\mel{x'}{\vb{p}}{p'}&=-i\hbar\pdv{x'}\ip{x'}{p'}\\
p'\ip{x'}{p'}&=-i\hbar\pdv{x'}\ip{x'}{p'}\\
\implies\ip{x'}{p'}&=Ne^{\frac{i}{\hbar}p'x'}\qq{where $N$ is an arbitrary constant}
\end{aligned}
\end{equation}
Let us calculate the normalization constant $N$. Note that
\begin{equation}
\begin{aligned}
\ip{x'}{x''}=\delta(x'-x'')=\mel{x'}{\mathbb{I}}{x''}=\int_{-\infty}^{\infty}\ip{x'}{p'}\ip{p'}{x''}
\end{aligned}
\end{equation}
Substituting $\ip{x'}{p'}=Ne^{\frac{i}{\hbar}p'x'}$
\begin{equation}
\begin{aligned}
\delta(x'-x'')&=\int_{-\infty}^{\infty}Ne^{\frac{i}{\hbar}p'x'}Ne^{-\frac{i}{\hbar}p'x''}\\
&=N^{2}\int_{-\infty}^{\infty}\dd{p'}e^{\frac{i}{\hbar}p'(x'-x'')}\\
&\quad\explain{When $x'-x''=0$, the integrand equals to 1. When $x'\neq x''$, the integrand oscillates}\\
&\quad\explain{$\delta(x)=\frac{1}{2\pi}\int_{-\infty}^{\infty}\dd{k}e^{ikx}$ is a known representation of the Dirac delta function}\\
\therefore\delta(x'-x'')&=N^{2}2\pi\hbar\delta(x'-x'')\\
\implies N&=\frac{1}{\sqrt{2\pi\hbar}}
\end{aligned}
\end{equation}
Thus, the momentum eigenfunction in the position representation is
\begin{equation}
\psi_{p'}(x')=\ip{x'}{p'}=\frac{1}{\sqrt{2\pi\hbar}}e^{\frac{i}{\hbar}p'x'}
\end{equation}
This refers to the usual plane wave solution. One can also obtain this solution by solving the Schr\"{o}dinger equation $\vb{H}\psi=E\psi$ and using a free Hamiltonian $\vb{H}=\frac{\vb{p}^{2}}{2m}=-\frac{\hbar^{2}}{2m}\pdv[2]{x}$.\\

Generalizing to 3D
\begin{equation}
\ip{\vec{x}\,'}{\vec{p}'}=\frac{1}{(2\pi\hbar)^{\frac{3}{2}}}e^{\frac{i}{\hbar}\vec{p}'\cdot\vec{x}\,'}
\end{equation}
Thus, we can convert between the momentum representation $\psi(p')$ and the position representation $\psi(x')$
\begin{equation}
\begin{aligned}
\psi(x')&=\ip{x'}{\psi}\\
&=\int_{-\infty}^{\infty}\dd{p'}\ip{x'}{p'}\ip{p'}{\psi}\\
&=\frac{1}{\sqrt{2\pi\hbar}}\int_{-\infty}^{\infty}\dd{p'}e^{\frac{i}{\hbar}p'x'}\psi(p')\\
\psi(p')&=\ip{p'}{\psi}\\
&=\int_{-\infty}^{\infty}\dd{x'}\ip{p'}{x'}\ip{x'}{\psi}\\
&=\frac{1}{\sqrt{2\pi\hbar}}\int_{-\infty}^{\infty}\dd{p'}e^{-\frac{i}{\hbar}p'x'}\psi(x')\\
\end{aligned}
\end{equation}
This conversion is equivalent to the Fourier transform.

\subsection{Time Dependence of Kets}
Let $\ket{\psi,t_{0};t}$ be the state that was equal to $\ket{\psi,t_{0};t}$ be the state that was equal to $\ket{\psi,t_{0}}\equiv\ket{\psi}$ at time $t_{0}$ which becomes a different state $\ket{\psi,t_{0}=t}$ at a later time $t>t_{0}$. The two kets are related by the time evolution operator $U(t,t_{0})$.
\begin{equation}
\ket{\psi,t_{0};t}=U(t,t_{0})\ket{\psi,t_{0}}
\end{equation}
Let us expand $\ket{\psi,t_{0}}$ and $\ket{\psi,t_{0};t}$ in terms of the eigenkets $\ket{a'}$ of some observable $\vb{A}$.
\begin{equation}
\begin{aligned}
\ket{\psi,t_{0}}&=\sum_{a'}c_{a'}(t_{0})\ket{a'}\\
\ket{\psi,t_{0};t}&=\sum_{a'}c_{a'}(t)\ket{a'}
\end{aligned}
\end{equation}
where $c_{a'}(t_{0})=\ip{a'}{\psi,t_{0}}$ and $c_{a'}(t)=\ip{a'}{\psi,t_{0};t}$. Thus
\begin{equation}
\begin{aligned}
\ip{\psi,t_{0}}&=\sum_{a',a''}c_{a''}^{*}(t_{0})c_{a'}(t_{0})\ip{a''}{a'}\\
&=\sum_{a',a''}c_{a''}^{*}(t_{0})c_{a'}(t_{0})\delta_{a''a'}\\
&=\sum_{a'}\abs{c_{a'}(t_{0})}^{2}\\
&=1\qq{by normalization}
\end{aligned}
\end{equation}
Analogously, $\ip{\psi,t_{0};t}=\sum_{a'}\abs{c_{a'}(t)}^{2}=1$. We interpret $\abs{c_{a'}(t_{0})}^{2}$ as the probability that a measurement of a physical system $\vb{A}$ gives $a'$.
\begin{equation}
\begin{aligned}
\ip{\psi,t_{0};t}&=\mel{\psi,t_{0}}{U^{\dagger}(t,t_{0})U(t,t_{0})}{\psi,t_{0}}\\
&=\ip{\psi,t_{0}}\\
\implies U^{\dagger}(t,t_{0})U(t,t_{0})&=\mathbb{I}\\
\implies U^{\dagger}(t,t_{0})&=U^{-1}(t,t_{0})
\end{aligned}
\end{equation}
Thus, $U(t,t_{0})$ is a unitary operator.\\

Likewise, by the composition property, for $t_{2}>t_{1}>t_{0}$
\begin{equation}
U(t_{2},t_{0})=U(t_{2},t_{1})U(t_{1},t_{0})
\end{equation}
Consider an infinitesimal time evolution operator $U(t_{0}+\dd{t},t_{0})$. Just as with the translation case, the unitary and composition requirements are satisfied if
\begin{equation}
U(t_{0}+\dd{t},t_{0})=\mathbb{I}-i\vb{\Omega}\dd{t}
\end{equation}
where $\vb{\Omega}=\vb{\Omega}^{\dagger}$ is a Hermitian operator.\\

Proving the unitary condition
\begin{equation}
\begin{aligned}
U^{\dagger}(t_{0}+\dd{t},t_{0})U(t_{0}+\dd{t},t_{0})&=\qty(\mathbb{I}+i\vb{\Omega}^{\dagger}\dd{t})\qty(\mathbb{I}-i\vb{\Omega}\dd{t})\\
&=\mathbb{I}+i\vb{\Omega}^{\dagger}\dd{t}-i\vb{\Omega}\dd{t}+\cancelto{0}{\order{\dd{t}^{2}}}\\
&=\mathbb{I}
\end{aligned}
\end{equation}

Proving the composition condition
\begin{equation}
\begin{aligned}
U(t_{0}+\dd{t_{1}}+\dd{t_{2}},t_{0}+\dd{t_{1}})U(t_{0}+\dd{t_{1}},t_{0})&=\qty(\mathbb{I}-i\vb{\Omega}\dd{t_{2}})\qty(\mathbb{I}-i\vb{\Omega}\dd{t_{1}})\\
&=\mathbb{I}-i\vb{\Omega}(\dd{t_{1}}+\dd{t_{2}})+\cancelto{0}{\order{\dd{t_{1}}\dd{t_{2}}}}\\
&=U(t_{0}+\dd{t_{1}}+\dd{t_{2}},t_{0})
\end{aligned}
\end{equation}

Since $\vb{\Omega}\dd{t}$ is dimensionless, by dimensional analysis, $\vb{\Omega}$ has dimensions of inverse time (or frequency). The energy and frequency of a system are related by $E=\hbar\omega$. Thus
\begin{equation}
\begin{aligned}
\vb{\Omega}&=\frac{\vb{H}}{\hbar}\\
\therefore U(t_{0}+\dd{t},t_{0})&=\mathbb{I}-\frac{i}{\hbar}\vb{H}\dd{t}
\end{aligned}
\end{equation}
Using this result, one can recover the Schr\"{o}dinger equation. Consider the difference of two time-evolution operators
\begin{equation}
\begin{aligned}
U(t+\dd{t},t_{0})-U(t,t_{0})&=U(t+\dd{t},t)U(t,t_{0})-U(t,t_{0})\\
&=\qty(\mathbb{I}-\frac{i}{\hbar}\vb{H}\dd{t})U(t,t_{0})-\mathbb{I}U(t,t_{0})\\
&=-\frac{i}{\hbar}\vb{H}\dd{t}U(t,t_{0})
\end{aligned}
\end{equation}
Dividing both sides by $\dd{t}$ and taking the limit as $\dd{t}$ going to 0
\begin{equation}
\begin{aligned}
\lim_{\dd{t}\rightarrow0}\frac{U(t+\dd{t},t_{0})-U(t,t_{0})}{\dd{t}}&=-\frac{i}{\hbar}\vb{H}U(t,t_{0})\\
i\hbar\pdv{t}U(t,t_{0})&=\vb{H}U(t,t_{0})\qq{(Schr\"{o}dinger equation for operators)}\\
i\hbar\pdv{t}U(t,t_{0})\ket{\psi,t_{0}}&=\vb{H}U(t,t_{0})\ket{\psi,t_{0}}\\
i\hbar\pdv{t}\ket{\psi,t_{0};t}&=\vb{H}\ket{\psi,t_{0}\psi,t_{0};t}\qq{(Schr\"{o}dinger equation for states)}
\end{aligned}
\end{equation}
Multiplying both sides on the left by $\bra{\vec{x}\,'}$
\begin{equation}
\begin{aligned}
i\hbar\pdv{t}\ip{\vec{x}\,'}{\psi,t_{0};t}&=\mel{\vec{x}\,'}{\vb{H}}{\psi,t_{0};t}\\
&=\int\dd[3]{x''}\mel{\vec{x}\,'}{\vb{H}}{\vec{x}\,''}\ip{\vec{x}\,''}{\psi,t_{0};t}\\
&\quad\explain{By resolution of identity}
\end{aligned}
\end{equation}
$\mel{\vec{x}\,'}{\vb{H}}{\vec{x}\,''}$ is the Hamiltonian in terms of the position basis, where
\begin{equation}
\begin{aligned}
\mel{\vec{x}\,'}{\vb{H}}{\vec{x}\,''}=-\frac{\hbar^{2}}{2m}\vec{\nabla}\delta(\vec{x}\,'-\vec{x}\,'')+\mel{\vec{x}\,'}{V(\vec{x})}{\vec{x}\,''}
\end{aligned}
\end{equation}
Let us solve for $U(t,t_{0})$ explicitly. We shall consider the Hamiltonian $\vb{H}$ to be time independent. Thus
\begin{equation}
\begin{aligned}
U(t,t_{0})&=e^{-\frac{i}{\hbar}\vb{H}(t-t_{0})}\\
&=\mathbb{I}-\frac{i}{\hbar}\vb{H}(t-t_{0})+\frac{1}{2\hbar^{2}}\qty[\vb{H}(t-t_{0})]^{2}+\order{(t-t_{0})^{3}}
\end{aligned}
\end{equation}
where $e^{\vb{A}}=\sum_{n=0}^{\infty}\frac{\vb{A}^{n}}{n!}$ is the operator exponential.\\

One finds that $U(t,t_{0})$ still satisfies Schr\"{o}dinger's equation. Likewise, the equation for $U(t,t_{0})$ is consistent with the equation $U(t+\dd{t},t_{0})=\mathbb{I}-\frac{i}{\hbar}\vb{H}\dd{t}$.\\

Recall the eigensystem of the Hamiltonian $\vb{H}$ is given by
\begin{equation}
\vb{H}\ket{a'}=E_{a'}\ket{a'}
\end{equation}
where $E_{a'}$ are the energy eigenvalues. Thus
\begin{equation}
\begin{aligned}
e^{-\frac{i}{\hbar}\vb{H}t}&=\mathbb{I}(e^{-\frac{i}{\hbar}\vb{H}t})\mathbb{I}\\
&=\sum_{a',a''}\op{a''}e^{-\frac{i}{\hbar}\vb{H}t}\op{a'}\\
&=\sum_{a',a''}\op{a''}e^{-\frac{i}{\hbar}E_{a'}t}\op{a'}\\
&\quad\explain{Since $e^{-\frac{i}{\hbar}\vb{H}t}$ is just a linear combination of powers of $\vb{H}$}\\
&=\sum_{a',a''}\ket{a''}e^{-\frac{i}{\hbar}E_{a'}t}\ip{a''}{a'}\bra{a'}\\
&=\sum_{a',a''}\ket{a''}e^{-\frac{i}{\hbar}E_{a'}t}\delta_{a'a''}\bra{a'}\\
&=\sum_{a'}\ket{a'}e^{-\frac{i}{\hbar}E_{a'}t}\bra{a'}
\end{aligned}
\end{equation}
Given this, one can explicitly find the time dependence of any state
\begin{equation}
\ket{\psi,t_{0}=0}=\sum_{a'}\ket{a'}\ip{a'}{\psi}=\sum_{a'}c_{a'}(0)\ket{a'}
\end{equation}
At some later time $t$
\begin{equation}
\begin{aligned}
\ket{\psi,t_{0}=0;t}&=U(t,0)\ket{\psi,t_{0}=0}\\
&=e^{-\frac{i}{\hbar}\vb{H}t}\ket{\psi,t_{0}=0}\\
&=\sum_{a'}e^{-\frac{i}{\hbar}E_{a'}t}\ket{a'}\ip{a'}{\psi,t_{0}=0}\\
&=\sum_{a'}c_{a'}(t)\ket{a'}
\end{aligned}
\end{equation}
Thus
\begin{equation}
c_{a'}(t)=c_{a'}(0)e^{-\frac{i}{\hbar}E_{a'}t}=\ip{a'}{\psi,t_{0}=0}e^{-\frac{i}{\hbar}E_{a'}t}
\end{equation}
The coefficients thus evolve in the same manner as $\ket{\psi}$.

\subsection{Different Pictures in Quantum Mechanics}
In the Schr\"{o}dinger picture, the states depend on time, while the operators do not. In contrast, the Heisenberg picture has the states be time-independent, while operators are time-dependent. To illustrate, consider some unitary operator $U$.
\begin{equation}
\ket{\psi}\rightarrow U\ket{\psi}\qq{($U$ transforms ket $\ket{\psi}$ into ket $U\ket{\psi}$)}
\end{equation}
Now consider an arbitrary Hermitian operator $\vb{A}$
\begin{equation}
\mel{\varphi}{\vb{A}}{\psi}\rightarrow\mel{\varphi}{U^{\dagger}\vb{A}U}{\psi}
\end{equation}
In the Schr\"{o}dinger picture, the unitary operator acts on the states
\begin{equation}
\mel{\varphi}{\vb{A}}{\psi}\rightarrow(\bra{\varphi}U^{\dagger})\vb{A}(U\ket{\psi})
\end{equation}
In the Heisenberg picture, the action is on the operator $\vb{A}$
\begin{equation}
\mel{\varphi}{\vb{A}}{\psi}\rightarrow\bra{\varphi}(U^{\dagger}\vb{A}U)\ket{\psi}
\end{equation}
Thus, instead of thinking about the transformation of kets, one can think of transforming operators
\begin{equation}
\vb{A}\rightarrow U^{\dagger}\vb{A}U
\end{equation}
while the kets remain unchanged.\\

Take $U(t)=e^{-\frac{i}{\hbar}\vb{H}t}$ to be the time-evolution operator. Define the Heisenberg picture operator as
\begin{equation}
\vb{A}^{(\trm{H})}(t)=U^{\dagger}(t)\vb{A}^{(\trm{S})}U(t)
\end{equation}
where $\vb{A}^{\trm{(S)}}$ is the time-independent Schr\"{o}dinger picture operator.\\

At $t=0$
\begin{equation}
\begin{aligned}
\vb{A}^{(\trm{H})}(0)&=U^{\dagger}(0)\vb{A}^{(\trm{S})}U(0)\\
&=\mathbb{I}\vb{A}^{\trm{(S)}}\mathbb{I}\\
&=\vb{A}^{\trm{(S)}}
\end{aligned}
\end{equation}
The kets in the Heisenberg picture do not change with time. Thus
\begin{equation}
\ket{\psi,t_{0}=0;t}_{\trm{H}}=\ket{\psi,t_{0}=0}
\end{equation}
In contrast, the Schr\"{o}dinger picture has kets of the form
\begin{equation}
\ket{\psi,t_{0}=0;t}_{\trm{S}}=U(t)\ket{\psi,t_{0}=0}
\end{equation}

Note that the expectation value of $\vb{A}$ is still the same in both pictures
\begin{equation}
\begin{aligned}
\tensor*[_{\trm{S}}]{\mel{\psi,t_{0}=0;t}{\vb{A}^{\trm{(S)}}}{\psi,t_{0}=0;t}}{_{\trm{S}}}&=\mel{\psi,t_{0}=0}{U^{\dagger}(t)A^{(\trm{S})}U(t)}{\psi,t_{0}=0}\\
&=\tensor*[_{\trm{H}}]{\mel{\psi,t_{0}=0}{\vb{A}^{\trm{(H)}}(t)}{\psi,t_{0}=0}}{_{\trm{H}}}
\end{aligned}
\end{equation}
Thus
\begin{equation}
\expval{\vb{A}^{\trm{(S)}}}=\expval{\vb{A}^{\trm{(H)}}}
\end{equation}
Let us derive an equation of motion for $\vb{A}^{\trm{(H)}}(t)$
\begin{equation}
\begin{aligned}
\vb{A}^{\trm{(H)}}(t)&=U^{\dagger}(t)\vb{A}^{(\trm{S})}U(t)=e^{\frac{i}{\hbar}\vb{H}t}\vb{A}^{\trm{(S)}}e^{-\frac{i}{\hbar}\vb{H}t}\\
\dv{t}\vb{A}^{\trm{(H)}}&=\frac{i}{\hbar}\vb{H}e^{\frac{i}{\hbar}\vb{H}t}\vb{A}^{\trm{(S)}}e^{-\frac{i}{\hbar}\vb{H}t}-\frac{i}{\hbar}e^{\frac{i}{\hbar}\vb{H}t}\vb{A}^{\trm{(S)}}\underbrace{e^{-\frac{i}{\hbar}\vb{H}t}\vb{H}}_{\mathclap{\trm{commuting operators}}}\\
&=\frac{i}{\hbar}\vb{H}\vb{A}^{\trm{(H)}}-\frac{i}{\hbar}\vb{A}^{\trm{(H)}}\vb{H}\\
&=\frac{i}{\hbar}\comm{\vb{H}}{\vb{A}^{\trm{(H)}}}
\end{aligned}
\end{equation}
Note that the Hamiltonian remains the same in both pictures
\begin{equation}
\begin{aligned}
e^{\frac{i}{\hbar}\vb{H}t}\vb{H}e^{-\frac{i}{\hbar}\vb{H}t}&=\vb{H}e^{\frac{i}{\hbar}\vb{H}t}e^{-\frac{i}{\hbar}\vb{H}t}\\
&=\vb{H}\mathbb{I}\\
&=\vb{H}
\end{aligned}
\end{equation}
This is a manifestation of conservation of energy.

\subsection{Conserved Quantities}
Let us apply the Heisenberg equation of motion to a free particle
\begin{equation}
\begin{aligned}
\vb{H}&=\frac{(\vec{\vb{p}})^{2}}{2m}\\
\therefore \dv{\vb{p}_{i}}{t}&=\frac{i}{\hbar}\comm{\vb{H}}{\vb{p}_{i}}=0
\end{aligned}
\end{equation}
Since $\vb{p}_{i}$ will commute with any function of $\vb{p}_{i}$, this expresses a conservation of momentum for a free particle.\\

In general, any operator $\vb{A}$ such that $\comm{\vb{H}}{\vb{A}}=0$ represents a conserved physical quantity of the system with Hamiltonian $\vb{H}$.\\

Now consider position operator $\vec{\vb{x}}$ which does not commute with $\vb{H}$.
\begin{equation}
\dv{\vb{x}_{i}}{t}=\frac{i}{\hbar}\comm{\vb{H}}{\vb{x}_{i}}
\end{equation}
Recall that $\comm{\vb{x}_{i}}{\vb{p}_{i}}=\vb{x}_{i}\vb{p}_{i}-\vb{p}_{i}\vb{x}_{i}=i\hbar$. also note that $(\vec{\vb{p}})^{2}=\sum_{i=1}^{3}\vb{p}_{i}^{2}$. Thus
\begin{equation}
\begin{aligned}
\comm{\vb{p}_{i}^{2}}{\vb{x}_{i}}&=\vb{p}_{i}^{2}\vb{x}_{i}-\vb{x}_{i}\vb{p}_{i}^{2}\\
&=\vb{p}_{i}^{2}\vb{x}_{i}-(i\hbar+\vb{p}_{i}\vb{x}_{i})\vb{p}_{i}\\
&=\vb{p}_{i}^{2}\vb{x}_{i}-i\hbar\vb{p}_{i}-\vb{p}_{i}\vb{x}_{i}\vb{p}_{i}\\
&=\cancel{\vb{p}_{i}^{2}\vb{x}_{i}}-i\hbar\vb{p}_{i}-\cancel{\vb{p}_{i}^{2}\vb{x}_{i}}-i\hbar\vb{p}_{i}\\
&=-2i\hbar\vb{p}_{i}
\end{aligned}
\end{equation}
Thus
\begin{equation}
\dv{\vb{x}_{i}}{t}=-\frac{i}{\hbar}\qty(\frac{1}{2m})(2i\hbar\vb{p}_{i})=\frac{\vb{p}_{i}}{m}
\end{equation}
By extension
\begin{equation}
\begin{aligned}
\comm{\vb{x}_{i}}{\vb{p}_{i}}&=i\hbar\\
\comm{\vb{x}_{i}^{2}}{\vb{p}_{i}}&=2i\hbar\vb{x}_{i}\\
&\vdots\\
\comm{\vb{x}_{i}^{n}}{\vb{p}_{i}}&=ni\hbar\vb{x}_{i}^{n-1}
\end{aligned}
\end{equation}
More generally, consider a particle interacting with some external potential
\begin{equation}
\vb{H}=\frac{(\vec{\vb{p}})^{2}}{2m}+V(\vec{\vb{x}})
\end{equation}
Thus
\begin{equation}
\dv{\vb{p}_{i}}{t}=\frac{i}{\hbar}\comm{\vb{H}}{\vb{p}_{i}}=\frac{i}{\hbar}\comm{V(\vec{\vb{x}})}{\vb{p}_{i}}
\end{equation}
Consider the action of the commutator on some state $\ket{\psi}$
\begin{equation}
\begin{aligned}
\comm{V(\vec{\vb{x}})}{\vb{p}_{i}}\ket{\psi}&=V(\vec{\vb{x}})\vb{p}_{i}\ket{\psi}-\vb{p}_{i}\qty(V(\vec{\vb{x}})\ket{\psi})\\
&=V(\vec{\vb{x}})\vb{p}_{i}\ket{\psi}-\qty(\vb{p}_{i}V(\vec{\vb{x}}))\ket{\psi}-V(\vec{\vb{x}})\qty(\vb{p}_{i}\ket{\psi})\\
&\quad\explain{Since $\vb{p}_{i}$ is a derivative in the position representation}\\
&=-\qty(\vb{p}_{i}V(\vec{\vb{x}}))\ket{\psi}\\
&=i\hbar\pdv{x_{i}}V(\vec{\vb{x}})\ket{\psi}
\end{aligned}
\end{equation}
Thus, $\dv{\vb{p}_{i}}{t}=-\pdv{x_{i}}V(\vec{\vb{x}})$, which is quantum mechanical analogue of Newton's second law. This yields Ehrenfest theorem.\\

Consider now the acceleraion
\begin{equation}
\begin{aligned}
\dv[2]{\vb{x}_{i}}{t}&=\frac{i}{\hbar}\comm{\vb{H}}{\dv{\vb{x}_{i}}{t}}\\
&=\frac{i}{\hbar}\comm{\vb{H}}{\frac{\vb{p}_{i}}{m}}\qq{since $\vb{x}_{i}$ still commutes with $V(\vec{\vb{x}})$}\\
&=\frac{i}{\hbar m}\comm{V(\vec{\vb{x}})}{\vb{p}_{i}}\qq{since $\vb{p}_{i}$ and $\vb{p}_{j}$ commute}\\
&=-\frac{1}{m}\pdv{x_{i}}V(\vec{\vb{x}})
\end{aligned}
\end{equation}
Thus
\begin{equation}
\dv[2]{\vec{\vb{x}}}{t}=-\frac{1}{m}\vec{\nabla}V(\vec{\vb{x}})
\end{equation}

\example{1}{Take $V(\vec{\vb{x}})=e\vec{E}\cdot\vec{\vb{x}}$ to be the potential due to an electric field $\vec{E}$ for a particle of charge $-e$.
\begin{equation}
\vec{F}=-\vec{\nabla}V(\vec{\vb{x}})=-e\vec{E}
\end{equation}
The expectation value of both sides is
\begin{equation}
m\dv[2]{\expval{\vec{\vb{x}}}}{t}=-e\vec{E}
\end{equation}
Thus, the average position of the particle will move according to a classical equation of motion.
}

Let us write the equation
\begin{equation}
i\hbar\pdv{t}\ket{\psi,t_{0};t}=\vb{H}\ket{\psi,t_{0};t}
\end{equation}
as a regular Schr\"{o}dinger equation in the position representation.\\

We denote $\psi(\vec{x}\,',t)$ as $\ip{\vec{x}\,'}{\psi,t_{0};t}$. Also, we consider $\vb{H}=\frac{(\vec{\vb{p}})^{2}}{2m}+V(\vec{\vb{x}})$. Multiplying both sides by $\bra{\vec{x}\,'}$
\begin{equation}
\begin{aligned}
i\hbar\pdv{t}\ip{\vec{x}\,'}{\psi,t_{0};t}&=\mel{\vec{x}\,'}{\vb{H}}{\psi,t_{0};t}\\
i\hbar\pdv{t}\psi(\vec{x}\,',t)&=\mel{\vec{x}\,'}{\frac{(\vec{\vb{p}})^{2}}{2m}}{\psi,t_{0};t}+\mel{\vec{x}\,'}{V(\vec{x}\,')}{\psi,t_{0};t}
\end{aligned}
\end{equation}
Note that $V(\vec{\vb{x}})\ket{\vec{x}\,'}=V(\vec{x}\,')\ket{\vec{x}\,'}$
\begin{equation}
\implies \mel{\vec{x}\,''}{V(\vec{\vb{x}})}{\vec{x}\,'}=V(\vec{x}\,')\ip{\vec{x}\,''}{\vec{x}\,'}=V(\vec{x}\,')\delta(\vec{x}\,''-\vec{x}\,')
\end{equation}
Earlier, we found that $\mel{\vec{x}\,'}{\vec{\vb{p}}}{\psi}=-i\hbar\vec{\nabla}'\ip{\vec{x}\,'}{\psi}$. Similarly
\begin{equation}
\begin{aligned}
(\vec{\vb{p}}\cdot\vec{\vb{p}})\ket{\psi}&=(-i\hbar\vec{\nabla})\cdot(-i\hbar\vec{\nabla})\ket{\psi}=-\hbar^{2}\nabla^{2}\ket{\psi}\\
\implies\mel{\vec{x}\,''}{\frac{(\vec{\vb{p}})^{2}}{2m}}{\psi}&=\frac{1}{2m}\int\dd[3]{x'}\ip{\vec{x}\,''}{\vec{x}\,'}(-\hbar^{2}\nabla'^{2})\ip{\vec{x}\,'}{\psi}\\
&=-\frac{\hbar^{2}}{2m}\nabla''^{2}\ip{\vec{x}\,''}{\psi}\\
\implies\mel{\vec{x}\,'}{\frac{(\vec{\vb{p}})^{2}}{2m}}{\psi}&=-\frac{\hbar^{2}}{2m}\nabla'^{2}\ip{\vec{x}\,'}{\psi}
\end{aligned}
\end{equation}
Note that
\begin{equation}
\bra{\vec{x}\,'}V(\vec{\vb{x}})=(V(\vec{\vb{x}})\ket{\vec{x}\,'})^{\dagger}=(V(\vec{x}\,')\ket{\vec{x}\,'})^{\dagger}=\bra{\vec{x}\,'}V(\vec{x}\,')
\end{equation}
Thus, Schr\"{o}dinger equation becomes
\begin{equation}
\begin{aligned}
i\hbar\pdv{t}\ip{\vec{x}\,'}{\psi,t_{0};t}&=-\frac{\hbar^{2}}{2m}\nabla'^{2}\ip{\vec{x}\,'}{\psi,t_{0};t}+V(\vec{x}\,')\ip{\vec{x}\,'}{\psi,t_{0};t}\\
i\hbar\pdv{t}\psi(\vec{x}\,',t)&=-\frac{\hbar^{2}}{2m}\nabla'^{2}\psi(\vec{x}\,',t)+V(\vec{x}\,')\psi(\vec{x}\,',t)
\end{aligned}
\end{equation}
This is the ordinary Schr\"{o}dinger equation for wavefunctions.\\

Let us derive the time-independent Schr\"{o}dinger equation. The time-dependence of an energy eigenstate is given by
\begin{equation}
\begin{aligned}
\ket{a',t_{0}=0;t}&=e^{-\frac{i}{\hbar}E_{a'}t}\ket{a',t_{0}=0}\\
&=e^{-\frac{i}{\hbar}E_{a'}t}\ket{a'}
\end{aligned}
\end{equation}
Thus, $\ip{\vec{x}\,'}{a',t_{0}=0;t}=e^{-\frac{i}{\hbar}E_{a'}t}\ip{\vec{x}\,'}{a'}$. Substituting into the original Schr\"{o}dinger equation
\begin{equation}
E_{a'}\ip{\vec{x}\,'}{a'}=-\frac{\hbar^{2}}{2m}\nabla'^{2}\ip{\vec{x}\,'}{a'}+V(\vec{x}\,')\ip{\vec{x}\,'}{a'}
\end{equation}
This is the time-independent Schr\"{o}dinger equation for stationary states of the Hamiltonian.

\newpage
\section{Rotations and Angular Momentum}
Consider the algebra of the rotations in 3D space. Suppose one has a Cartesian coordinate system $\{x,y,z\}$. To be consistent, we shall use the active view of rotations (the physical system is rotated while the coordinate system is kept fixed).\\

Consider rotation the system $\vec{r}$ about the $z$-axis to the vector $\vec{r}\,'$ by an angle $\varphi$. We take position $\varphi$ to be a right-handed rotation around the $z$-axis. Note the $z$-component remains constant.

\begin{center}
\begin{tikzpicture}
	\draw[->] (0,0) -- (4,0) node[right] {$\hat{x}$};
	\draw[->] (0,0) -- (0,4) node[above] {$\hat{y}$};
	\draw[->] (0,0) -- (3,2.646) node[above] {$\vec{r}\,'$};
	\draw[->] (0,0) -- (3.872,1) node[right] {$\vec{r}$};
	\draw[->] (0,0) -- (-1,3.872) node[left] {$\hat{z}\times\vec{r}$};
	\draw[fill=white] (0,0) circle (4pt);
	\fill (0,0) circle (2pt) node[below left] {$\hat{z}$};
	\draw (0.9682,0.2500) arc (14.4775:41.4096:1) node[xshift=0.2cm,right] {$\varphi$};
\end{tikzpicture}
\end{center}
Let the vector perpendicular to $\vec{r}$ be denoted $\hat{z}\times\vec{r}$. One can then write the vector $\vec{r}\,'$ as
\begin{equation}
\begin{aligned}
\vec{r}\,'&=\vec{r}\cos\varphi+(\hat{z}\times\vec{r})\sin\varphi\\
&=(x\hat{x}+y\hat{y})\cos\varphi+[\hat{z}\times(x\hat{x}+y\hat{y})]\sin\varphi\\
&=(x\hat{x}+y\hat{y})\cos\varphi+(x\hat{y}-y\hat{x})\sin\varphi\\
&=(x\cos\varphi-y\sin\varphi)\hat{x}+(y\cos\varphi+x\sin\varphi)\hat{y}
\end{aligned}
\end{equation}
Thus, $x'=x\cos\varphi-y\sin\varphi$, $y'=x\sin\varphi+y\cos\varphi$, and $z'=z$. Moreover
\begin{equation}
\begin{aligned}
\vec{r}\,'&=R_{z}(\varphi)\vec{r}\\
\mqty[x'\\y'\\z']&=\mqty[\cos\varphi & -\sin\varphi & 0\\ \sin\varphi & \cos\varphi & 0 \\ 0 & 0 & 1]\mqty[x\\y\\z]
\end{aligned}
\end{equation}
$R_{z}(\varphi)$ is known as the rotation matrix about the $z$-axis. Analogously, one can find for $R_{x}(\varphi)$ and $R_{y}(\varphi)$
\begin{equation}
\begin{aligned}
R_{x}(\varphi)&=\mqty[1 & 0 & 0 \\ 0 & \cos\varphi & -\sin\varphi \\ 0 & \sin\varphi & \cos\varphi]\\
R_{y}(\varphi)&=\mqty[\cos\varphi & 0 & -\sin\varphi \\ 0 & 1 & 0 \\ \sin\varphi & 0 & \cos\varphi]
\end{aligned}
\end{equation}
Consider the transpose of a rotation matrix $R_{z}^{T}(\varphi)$.
\begin{equation}
\begin{aligned}
R_{z}^{T}(\varphi)&=\mqty[\cos\varphi & \sin\varphi & 0\\ -\sin\varphi & \cos\varphi & 0 \\ 0 & 0 & 1]\\
R_{z}(\varphi)R_{z}^{T}(\varphi)&=\mqty[\cos\varphi & -\sin\varphi & 0\\ \sin\varphi & \cos\varphi & 0 \\ 0 & 0 & 1]\mqty[\cos\varphi & \sin\varphi & 0\\ -\sin\varphi & \cos\varphi & 0 \\ 0 & 0 & 1]=\mathbb{I}\\
\implies R_{z}^{T}(\varphi)&=R_{z}^{-1}(\varphi)=R_{z}(-\varphi)
\end{aligned}
\end{equation}
One can find analogous results for $R_{x}(\varphi)$ and $R_{y}(\varphi)$. Thus, rotation matrices are orthogonal. Rotation matrices are always represented by orthogonal matrices.\\

Note that in general
\begin{equation}
R_{i}(\varphi_{1})R_{j}(\varphi_{2})\neq R_{j}(\varphi_{1})R_{i}(\varphi_{2}),\quad i,j\in\{x,y,z\},\quad i\neq j
\end{equation}
Thus
\begin{equation}
\comm{R_{i}}{R_{j}}\neq 0\qq{for $i\neq j$, $i,j\in\{x,y,z\}$}
\end{equation}
Consider infinitesimal rotations $\varphi=\epsilon\ll1$ and ignore $\order{\epsilon^{3}}$.
\begin{equation}
\begin{aligned}
R_{x}(\epsilon)&=\mqty[1 & 0 & 0 \\ 0 & 1-\frac{\epsilon^{2}}{2} & -\epsilon \\ 0 & \epsilon & 1-\frac{\epsilon^{2}}{2}]\\
R_{y}(\epsilon)&=\mqty[1-\frac{\epsilon^{2}}{2} & 0 & -\epsilon \\ 0 & 1 & 0 \\ \epsilon & 0 & 1-\frac{\epsilon^{2}}{2}]\\
R_{z}(\epsilon)&=\mqty[1-\frac{\epsilon^{2}}{2} & -\epsilon & 0 \\ \epsilon & 1-\frac{\epsilon^{2}}{2} & 0 \\ 0 & 0 & 1]
\end{aligned}
\end{equation}
Consider commutation relations between rotations about different axes. Note that rotations do not commute while transformations do commute ($\vb{T}(\vec{x})\vb{T}(\vec{y})=\vb{T}(\vec{y})\vb{T}(\vec{x})$).
\begin{center}
\begin{tikzpicture}
	\draw[->,shorten >=1mm,shorten <=1mm] (0,0) -- node[below] {$\vb{T}(\vec{x})$} (3,0);
	\draw[->,shorten >=1mm,shorten <=1mm] (3,0) -- node[right] {$\vb{T}(\vec{y})$} (3,3);
	\draw[->,shorten >=1mm,shorten <=1mm] (0,0) -- node[left] {$\vb{T}(\vec{y})$} (0,3);
	\draw[->,shorten >=1mm,shorten <=1mm] (0,3) -- node[above] {$\vb{T}(\vec{x})$} (3,3);
\end{tikzpicture}
\end{center}
\begin{center}
\begin{tikzpicture}
	\draw[->] (0,0,0) -- (2,0,0) node[right] {$x$};
	\draw[->] (0,0,0) -- (0,2,0) node[above] {$y$};
	\draw[->] (0,0,0) -- (0,0,2) node[below left] {$z$};
	\draw[->, ultra thick,red] (0,0,0) -- (0,1.5,0) node[above right] {$\vec{r}$};
	\draw[->] (3,1,0) -- node[above] {$R_{x}\qty(\frac{\pi}{2})$} (5,1,0);
	\draw[->] (6,0,0) -- (8,0,0) node[right] {$x$};
	\draw[->] (6,0,0) -- (6,2,0) node[above] {$y$};
	\draw[->] (6,0,0) -- (6,0,2) node[below left] {$z$};
	\draw[->, ultra thick,red] (6,0,0) -- (6,0,1.5) node[above left] {$\vec{r}$};
	\draw[->] (9,1,0) -- node[above] {$R_{y}\qty(\frac{\pi}{2})$} (11,1,0);
	\draw[->] (12,0,0) -- (14,0,0) node[right] {$x$};
	\draw[->] (12,0,0) -- (12,2,0) node[above] {$y$};
	\draw[->] (12,0,0) -- (12,0,2) node[below left] {$z$};
	\draw[->, ultra thick,red] (12,0,0) -- (13.5,0,0) node[above right] {$\vec{r}$};
	\draw[->] (0,-4,0) -- (2,-4,0) node[right] {$x$};
	\draw[->] (0,-4,0) -- (0,-2,0) node[above] {$y$};
	\draw[->] (0,-4,0) -- (0,-4,2) node[below left] {$z$};
	\draw[->, ultra thick,red] (0,-4,0) -- (0,-2.5,0) node[above right] {$\vec{r}$};
	\draw[->] (3,-3,0) -- node[above] {$R_{y}\qty(\frac{\pi}{2})$} (5,-3,0);
	\draw[->] (6,-4,0) -- (8,-4,0) node[right] {$x$};
	\draw[->] (6,-4,0) -- (6,-2,0) node[above] {$y$};
	\draw[->] (6,-4,0) -- (6,-4,2) node[below left] {$z$};
	\draw[->, ultra thick,red] (12,-4,0) -- (12,-2.5,0) node[above right] {$\vec{r}$};
	\draw[->] (9,-3,0) -- node[above] {$R_{x}\qty(\frac{\pi}{2})$} (11,-3,0);
	\draw[->] (12,-4,0) -- (14,-4,0) node[right] {$x$};
	\draw[->] (12,-4,0) -- (12,-2,0) node[above] {$y$};
	\draw[->] (12,-4,0) -- (12,-4,2) node[below left] {$z$};
	\draw[->, ultra thick,red] (12,-4,0) -- (13.5,-4,0) node[above left] {$\vec{r}$};
\end{tikzpicture}
\end{center}
\begin{equation}
\begin{aligned}
R_{x}(\epsilon)R_{y}(\epsilon)&=\ldots=\mqty[1-\frac{\epsilon^{2}}{2} & 0 & \epsilon \\ \epsilon^{2} & 1-\frac{\epsilon^{2}}{2} & -\epsilon \\ -\epsilon & \epsilon & 1-\epsilon^{2}]\\
R_{y}(\epsilon)R_{x}(\epsilon)&=\ldots=\mqty[1-\frac{\epsilon^{2}}{2} & \epsilon^{2} & \epsilon \\ 0 & 1-\frac{\epsilon^{2}}{2} & -\epsilon \\ -\epsilon & \epsilon & 1-\epsilon^{2}]\\
\comm{R_{x}(\epsilon)}{R_{y}(\epsilon)}&=\mqty[0 & -\epsilon^{2} & 0\\ \epsilon^{2} & 0 & 0\\ 0 & 0 & 0]=R_{z}(\epsilon^{2})-\mathbb{I} \qq{to order $\epsilon^{2}$}
\end{aligned}
\end{equation}
Thus, $R_{x}(\epsilon)$ and $R_{y}(\epsilon)$ commute to first order, but not to second order.

\subsection{Rotations in Quantum Mechanics}
Groups in which the elements commute, like translations are \ul{abelian}. Groups with non-commuting elements are \ul{non-abelian}.\\

Let us associate a rotation operator $D(R)$ to rotation matrix $R$.
\begin{equation}
\ket{\psi}_{R}=D(R)\ket{\psi}
\end{equation}
$\ket{\psi}_{R}$ is the rotated version of $\ket{\psi}$. If $R$ represents the rotation b an infinitesimally small angle $\dd{\varphi}$ about an axis $\hat{n}$, then by analogy with the infinitesimal translations and infinitesimal time-evolution, $D(R)$ should be of the form
\begin{equation}
D(\hat{n},\dd{\varphi})=\mathbb{I}-\frac{i}{\hbar}(\vec{\vb{J}}\cdot\hat{n})\dd{\varphi}
\end{equation}
where $\vec{\vb{J}}$ is a Hermitian operator and has dimensions of angular momentum. We say $\vec{\vb{J}}$ is the generator of rotations.\\

We find again for finite rotations
\begin{equation}
D(\hat{n},\varphi)=\lim_{N\rightarrow\infty}\qty[\mathbb{I}-\frac{i}{\hbar}(\vec{\vb{J}}\cdot\hat{n})\frac{\varphi}{N}]^{N}=e^{-\frac{i}{\hbar}\vec{\vb{J}}\cdot\hat{n}\varphi}
\end{equation}
$D(R)$ must have the same properties as the rotation matrices themselves. If $R_{3}=R_{1}R_{2}$, then $D(R_{3})=D(R_{1})D(R_{2})$. This means we can translate the commutation relations between the matrices themselves to commutation relations to the corresponding $D$ operators.
\begin{equation}
\therefore\comm{D_{x}(\epsilon)}{D_{y}(\epsilon)}=D_{z}(\epsilon^{2})-\mathbb{I}
\end{equation}
To demonstrate
\begin{equation}
\begin{aligned}
D_{x}(\epsilon)&=\mathbb{I}-\frac{i}{\hbar}\vb{J}_{x}\epsilon-\frac{1}{2\hbar^{2}}\vb{J}_{x}^{2}\epsilon^{2}+\order{\epsilon^{3}}\\
D_{y}(\epsilon)&=\mathbb{I}-\frac{i}{\hbar}\vb{J}_{y}\epsilon-\frac{1}{2\hbar^{2}}\vb{J}_{y}^{2}\epsilon^{2}+\order{\epsilon^{3}}\\
D_{z}(\epsilon)&=\mathbb{I}-\frac{i}{\hbar}J_{z}\epsilon^{2}+\order{\epsilon^{3}}
\end{aligned}
\end{equation}
Up to second order of $\epsilon$
\begin{equation}
\begin{aligned}
\comm{D_{x}(\epsilon)}{D_{y}(\epsilon)}&=D_{x}(\epsilon)D_{y}(\epsilon)-D_{y}(\epsilon)D_{x}(\epsilon)\\
&=\qty(\mathbb{I}-\frac{i}{\hbar}\vb{J}_{x}\epsilon-\frac{1}{2\hbar^{2}}\vb{J}_{x}^{2}\epsilon^{2})\qty(\mathbb{I}-\frac{i}{\hbar}\vb{J}_{y}\epsilon-\frac{1}{2\hbar^{2}}\vb{J}_{y}^{2}\epsilon^{2})\\
&\quad-\qty(\mathbb{I}-\frac{i}{\hbar}\vb{J}_{y}\epsilon-\frac{1}{2\hbar^{2}}\vb{J}_{y}^{2}\epsilon^{2})\qty(\mathbb{I}-\frac{i}{\hbar}\vb{J}_{x}\epsilon-\frac{1}{2\hbar^{2}}\vb{J}_{x}^{2}\epsilon^{2})\\
&=\cancel{\mathbb{I}}-\cancel{\frac{i}{\hbar}\vb{J}_{y}\epsilon}-\cancel{\frac{1}{2\hbar^{2}}\vb{J}_{y}^{2}\epsilon^{2}}-\cancel{\frac{i}{\hbar}\vb{J}_{x}\epsilon}-\cancel{\frac{1}{2\hbar^{2}}\vb{J}_{x}^{2}\epsilon^{2}}-\frac{1}{\hbar^{2}}\vb{J}_{x}\vb{J}_{y}\epsilon^{2}\\
&\quad-\cancel{\mathbb{I}}+\cancel{\frac{i}{\hbar}\vb{J}_{x}\epsilon}+\cancel{\frac{1}{2\hbar^{2}}\vb{J}_{x}^{2}\epsilon^{2}}+\cancel{\frac{i}{\hbar}\vb{J}_{y}\epsilon}+\cancel{\frac{1}{2\hbar^{2}}\vb{J}_{x}^{2}\epsilon^{2}}+\frac{1}{\hbar^{2}}\vb{J}_{y}\vb{J}_{x}\epsilon^{2}\\
&=-\frac{1}{\hbar^{2}}\comm{\vb{J}_{x}}{\vb{J}_{y}}\epsilon^{2}\\
&=D_{z}(\epsilon^{2})-\mathbb{I}\\
&=-\frac{i}{\hbar}\vb{J}_{z}\epsilon^{2}\\
\implies \comm{\vb{J}_{x}}{\vb{J}_{y}}&=i\hbar\vb{J}_{z}
\end{aligned}
\end{equation}
In general, $\comm{\vb{J}_{i}}{\vb{J}_{j}}=i\hbar\varepsilon_{ijk}\vb{J}_{k}$, where $\varepsilon_{ijk}$ is the fully antisymmetric tensor.
\begin{equation}
\begin{aligned}
\varepsilon_{xyz}&=\varepsilon_{yzx}=\varepsilon_{zxy}=1\\
\varepsilon_{xzy}&=\varepsilon_{zyx}=\varepsilon_{yxz}=1
\end{aligned}
\end{equation}
Groups of rotations in 3D are non-abelian since the generators of rotation do not commute. Groups of translations are abelian since their generators of translations commute.\\

When the generators of infinitesimal transformations do not commute, the corresponding group of operators is non-abelian. When they do commute, the corresponding group is abelian.

\subsection{Spin-$\frac{1}{2}$ Operators}
The simplest example of angular momentum is a spin-$\frac{1}{2}$ system. The spin operators $\vb{S}_{\hat{n}}$ satisfy the same commutation relation as angular momentum
\begin{equation}
\comm{\vb{S}_{i}}{\vb{S}_{j}}=i\hbar\varepsilon_{ijk}\vb{S}_{k}
\end{equation}
where $\vec{\vb{S}}=\frac{\hbar}{2}\vec{\sigma}=\frac{\hbar}{2}(\sigma_{x},\sigma_{y},\sigma_{z})$, with $\vec{\sigma}=(\sigma_{x},\sigma_{y},\sigma_{z})$ being the Pauli matrices.
\begin{equation}
\begin{aligned}
\sigma_{x}&=\mqty[0 & 1\\ 1&0]\\
\sigma_{y}&=\mqty[0 & -i\\ i & 0]\\
\sigma_{z}&=\mqty[1 & 0\\ 0&-1]
\end{aligned}
\end{equation}
Note that the Pauli matrices have the property
\begin{equation}
\sigma_{i}^{2}=\mathbb{I},\quad i\in\{x,y,z\}
\end{equation}
The spin operators can be written as
\begin{equation}
\vb{S}_{i}=\frac{\hbar}{2}\sum_{a,b}\ket{a}\sigma_{i_{ab}}\bra{b}
\end{equation}
where $\ket{a},\ket{b}\in\{\ket{\up},\ket{\dn}\}$. Thus, expanded in Dirac notation
\begin{equation}
\begin{aligned}
\vb{S}_{x}&=\frac{\hbar}{2}\qty(\op{\up}{\dn}+\op{\dn}{\up})\\
\vb{S}_{y}&=i\frac{\hbar}{2}\qty(-\op{\up}{\dn}+\op{\dn}{\up})\\
\vb{S}_{z}&=\frac{\hbar}{2}\qty(\op{\up}{\up}-\op{\dn}{\dn})
\end{aligned}
\end{equation}
where $\ket{\up}$ and $\ket{\dn}$ are eigenkets of $\vb{S}_{z}$. Note that
\begin{equation}
\begin{aligned}
\vb{S}_{z}\ket{\up}&=\frac{\hbar}{2}\ket{\up}\\
\vb{S}_{z}\ket{\dn}&=-\frac{\hbar}{2}\ket{\dn}
\end{aligned}
\end{equation}

\subsection{Rotation of Operators}
Consider a rotation about the $z$-axis by angle $\varphi$.
\begin{equation}
D_{z}(\varphi)=e^{-\frac{i}{\hbar}\vb{S}_{z}\varphi}
\end{equation}
The effect of this rotation on $\expval{\vb{S}_{x}}$ is
\begin{equation}
\begin{aligned}
\expval{\vb{S}_{x}}_{R}&=\tensor*[_{R}]{\mel{\psi}{\vb{S}_{x}}{\psi}}{_{R}}\\
&=\mel{\psi}{D_{z}^{\dagger}(\varphi)\vb{S}_{x}D_{z}(\varphi)}{\psi}
\end{aligned}
\end{equation}
Moreover
\begin{equation}
\begin{aligned}
D_{z}^{\dagger}(\varphi)\vb{S}_{x}D_{z}(\varphi)&=e^{\frac{i}{\hbar}\vb{S}_{z}\varphi}\vb{S}_{x}e^{-\frac{i}{\hbar}\vb{S}_{z}\varphi}\\
&=e^{\frac{i}{2}\sigma_{z}\varphi}\vb{S}_{x}e^{-\frac{i}{2}\sigma_{z}\varphi}\qq{since $\vec{\vb{S}}=\vec{\sigma}$}
\end{aligned}
\end{equation}
Performing a Taylor expansion
\begin{equation}
\begin{aligned}
D_{z}(\varphi)&=e^{-\frac{i}{2}\sigma_{z}\varphi}\\
&=\sum_{n=0}^{\infty}\frac{1}{n!}\qty(-\frac{i}{2})^{n}\sigma_{z}^{n}\varphi^{n}\\
&=\sum_{n=0}^{\infty}\frac{1}{(2n)!}\qty(-\frac{i\varphi}{2})^{2n}\mathbb{I}^{n}+\sum_{n=0}^{\infty}\frac{1}{(2n+1)!}\qty(-\frac{i\varphi}{2})^{2n+1}\mathbb{I}^{n}\sigma_{z}\\
&\quad\explain{Since $\sigma_{z}^{2}=\mathbb{I}$}\\
&=\sum_{n=0}^{\infty}\frac{(-1)^{n}}{(2n)!}\qty(\frac{\varphi}{2})^{2n}\mathbb{I}-i\sum_{n=0}^{\infty}\frac{(-1)^{n}}{(2n+1)!}\qty(\frac{\varphi}{2})^{2n+1}\sigma_{z}\\
&=\cos\qty(\frac{\varphi}{2})\mathbb{I}-i\sigma_{z}\sin\qty(\frac{\varphi}{2})
\end{aligned}
\end{equation}
This reasoning holds for any Pauli matrix. Thus
\begin{equation}
\begin{aligned}
D_{\hat{n}}(\varphi)=e^{-\frac{i}{2}\vec{\sigma}\cdot\hat{n}\varphi}=\cos\qty(\frac{\varphi}{2})\mathbb{I}-i\sin\qty(\frac{\varphi}{2})\vec{\sigma}\cdot\hat{n}
\end{aligned}
\end{equation}
Thus
\begin{equation}
\begin{aligned}
D_{z}^{\dagger}(\varphi)\vb{S}_{x}D_{z}(\varphi)&=\qty(\cos\qty(\frac{\varphi}{2})\mathbb{I}-i\sin\qty(\frac{\varphi}{2})\sigma_{z})\vb{S}_{x}\qty(\cos\qty(\frac{\varphi}{2})\mathbb{I}-i\sin\qty(\frac{\varphi}{2})\sigma_{z})\\
&=\frac{\hbar}{2}\qty[\cos^{2}\qty(\frac{\varphi}{2})\sigma_{x}+i\cos\qty(\frac{\varphi}{2})\sin\qty(\frac{\varphi}{2})\comm{\sigma_{z}}{\sigma_{x}}+\sin^{2}\qty(\frac{\varphi}{2})\sigma_{z}\sigma_{x}\sigma_{z}]
\end{aligned}
\end{equation}
From the commutation relation for $\vb{S}_{\hat{n}}$
\begin{equation}
\frac{\hbar^{2}}{4}\comm{\sigma_{i}}{\sigma_{j}}=i\hbar\qty(\frac{\hbar}{2})\varepsilon_{ijk}\sigma_{k}
\end{equation}
Thus, $\comm{\sigma_{i}}{\sigma_{j}}=2i\varepsilon_{ijk}\sigma_{k}$. By hand
\begin{equation}
\begin{aligned}
\sigma_{x}\sigma_{z}&=\ldots=-i\sigma_{y}\\
\sigma_{z}\sigma_{x}&=\ldots=i\sigma_{y}\\
\implies \sigma_{x}\sigma_{z}&=-\sigma_{z}\sigma_{x}
\end{aligned}
\end{equation}
Therefore
\begin{equation}
\begin{aligned}
D_{z}^{\dagger}(\varphi)\vb{S}_{x}D_{z}(\varphi)&=\frac{\hbar}{2}\qty[\cos^{2}\qty(\frac{\varphi}{2})\sigma_{x}+\frac{i}{2}\sin\qty(\varphi)\comm{\sigma_{z}}{\sigma_{x}}-\sin^{2}\qty(\frac{\varphi}{2})\sigma_{z}^{2}\sigma_{x}]\\
&=\frac{\hbar}{2}\qty[\qty(\cos^{2}\qty(\frac{\varphi}{2})-\sin^{2}\qty(\frac{\varphi}{2}))\sigma_{x}+\frac{i}{2}\sin\qty(\varphi)\comm{\sigma_{z}}{\sigma_{x}}]\\
&=\frac{\hbar}{2}\qty(\cos(\varphi)\sigma_{x}-\sin(\varphi)\sigma_{y})\\
&=\cos(\varphi)\vb{S}_{x}-\sin(\varphi)\vb{S}_{y}
\end{aligned}
\end{equation}
as expected from the rotation matrix $R_{z}(\varphi)$. Thus
\begin{equation}
\expval{\vb{S}_{x}}_{R}=\expval{\vb{S}_{x}}\cos\varphi-\expval{\vb{S}_{y}}\sin\varphi
\end{equation}
Both the operator $\vb{S}$ and its expectation values transform under rotation as an ordinary vector under a rotation matrix.

\subsection{Rotation of Kets}
In Dirac notation, any ket that represents a spin-$\frac{1}{2}$ system can be written as a linear combination of the eigenstates $\{\ket{\up},\ket{\dn}\}$ of $\vb{S}_{z}$.\\

These kets may alternatively be represented as $z$-component vectors (spinors)
\begin{equation}
\begin{aligned}
\ket{\up}&=\mqty[1 \\ 0],\quad &\ket{\dn}&=\mqty[0\\1]\\
\sigma_{z}\ket{\up}&=\mqty[1 & 0\\0 & -1]\mqty[1\\0]=\mqty[1\\0],\quad &\sigma_{z}\ket{\dn}&=\mqty[1 & 0\\0 & -1]\mqty[0\\1]=-\mqty[0\\1]
\end{aligned}
\end{equation}
For an arbitrary state $\ket{z}$
\begin{equation}
\begin{aligned}
\ket{z}&=z_{\up}\ket{\up}+z_{\dn}\ket{\dn}=\mqty[z_{\up}\\z_{\dn}]
\end{aligned}
\end{equation}
$z_{\up}$ and $z_{\dn}$ are complex amplitudes, such that they normalize $\ket{z}$.
\begin{equation}
\ip{z}=\abs{z_{\up}}^{2}+\abs{z_{\dn}}^{2}=1\qq{by the normalization condition}
\end{equation}
$\abs{z_{\up}}^{2}$ and $\abs{z_{\dn}}^{2}$ are the probabilities to find the system $\ket{z}$ in eigenstate $\ket{\up}$ and $\ket{\dn}$ respectively after measurement.\\

Let us find the spinor that is an eigenstate of $\vec{\sigma}\cdot\hat{n}$
\begin{equation}
\vec{\sigma}\cdot\hat{n}\ket{z}=\ket{z}\qq{since eigenvalues of $\vec{\sigma}\cdot\hat{n}$ are $\pm1$}
\end{equation}
where $\hat{n}=(\sin\theta\cos\phi,\sin\theta\sin\phi,\cos\theta)$.\\

Suppose we start from $\hat{n}$ along the $z$-axis and want to rotate it to a general direction specified by $\theta$ and $\varphi$. To accomplish this, one first rotates by angle $\theta$ around the $y$-axis then by angle $\varphi$ around the $z$-axis. Thus
\begin{equation}
\begin{aligned}
\ket{z}&=e^{-\frac{i}{2}\sigma_{z}\varphi}e^{-\frac{i}{2}\sigma_{y}\theta}\ket{\up}\\
&=\mqty[\cos\qty(\frac{\varphi}{2})-i\sin\qty(\frac{\varphi}{2}) & 0\\0 & \cos\qty(\frac{\varphi}{2})+i\sin\qty(\frac{\varphi}{2})]\mqty[\cos\qty(\frac{\theta}{2}) & -\sin\qty(\frac{\theta}{2}) \\ \sin\qty(\frac{\theta}{2}) & \cos\qty(\frac{\theta}{2})]\mqty[1\\0]\\
&=\mqty[e^{-i\frac{\varphi}{2}}\cos\qty(\frac{\theta}{2}) \\ e^{i\frac{\varphi}{2}}\sin\qty(\frac{\theta}{2})]
\end{aligned}
\end{equation}
Consider the expectation values of $\vec{\vb{S}}$ now
\begin{equation}
\begin{aligned}
\mel{z}{\vb{S}_{x}}{z}&=\frac{\hbar}{2}\mel{z}{\sigma_{x}}{z}\\
&=\frac{\hbar}{2}\sum_{a,b}z_{a}^{*}\sigma_{x_{ab}}z_{b}\quad a,b\in\{\ket{\up},\ket{\dn}\}\\
&=\frac{\hbar}{2}\qty(z_{\up}^{*}z_{\dn}+z_{\dn}^{*}z_{\up})\\
&=\frac{\hbar}{2}\sin\theta\cos\varphi\\
&=\frac{\hbar}{2}n_{x}\\
\mel{z}{\vb{S}_{y}}{z}&=\ldots\\
&=\frac{\hbar}{2}\sin\theta\sin\varphi\\
&=\frac{\hbar}{2}n_{y}\\
\mel{z}{\vb{S}_{z}}{z}&=\ldots\\
&=\frac{\hbar}{2}\cos\theta\\
&=\frac{\hbar}{2}n_{z}
\end{aligned}
\end{equation}
Thus, $\mel{z}{\vec{\vb{S}}}{z}=\frac{\hbar}{2}\hat{n}$, where $\hat{n}=z_{a}^{*}\vec{\sigma}_{ab}z_{b}$ is another way to represent a unit vector in terms of two complex numbers satisfying the condition $\abs{z_{\up}}^{2}+\abs{z_{\dn}}^{2}=1$. Note that we shall ignore Euler angles in this course.

\subsection{Theory of Angular Momentum}
Consider the square of the angular momentum operator $\vec{\vb{J}}$.
\begin{equation}
\vec{\vb{J}}^{2}=\vb{J}_{x}^{2}+\vb{J}_{y}^{2}+\vb{J}_{z}^{2}
\end{equation}
Consider its commutator with $\vb{J}_{z}$
\begin{equation}
\begin{aligned}
\comm{\vec{\vb{J}}^{2}}{\vb{J}_{z}}&=\comm{\vb{J}_{x}^{2}+\vb{J}_{y}^{2}+\vb{J}_{z}^{2}}{\vb{J}_{z}}\\
&=\comm{\vb{J}_{x}^{2}+\vb{J}_{y}^{2}}{\vb{J}_{z}}\\
&\quad\explain{Since $\vb{J}_{z}$ commutes with powers of itself}\\
&=\vb{J}_{x}^{2}\vb{J}_{z}-\vb{J}_{z}\vb{J}_{x}^{2}+\vb{J}_{y}^{2}\vb{J}_{z}-\vb{J}_{z}\vb{J}_{y}^{2}
\end{aligned}
\end{equation}
Note that
\begin{equation}
\begin{aligned}
\vb{J}_{x}\comm{\vb{J}_{x}}{\vb{J}_{z}}+\comm{\vb{J}_{x}}{\vb{J}_{z}}\vb{J}_{x}&=\vb{J}_{x}(\vb{J}_{x}\vb{J}_{z}-\vb{J}_{z}\vb{J}_{x})+(\vb{J}_{x}\vb{J}_{z}-\vb{J}_{z}\vb{J}_{x})\vb{J}_{x}\\
&=\vb{J}_{x}^{2}\vb{J}_{z}-\vb{J}_{x}\vb{J}_{z}\vb{J}_{x}+\vb{J}_{x}\vb{J}_{z}\vb{J}_{x}-\vb{J}_{z}\vb{J}_{x}^{2}\\
&=\vb{J}_{x}^{2}\vb{J}_{z}-\vb{J}_{z}\vb{J}_{x}^{2}\\
\therefore \comm{\vec{\vb{J}}^{2}}{\vb{J}_{z}}&=\vb{J}_{x}\comm{\vb{J}_{x}}{\vb{J}_{z}}+\comm{\vb{J}_{x}}{\vb{J}_{z}}\vb{J}_{x}+\vb{J}_{y}\comm{\vb{J}_{x}}{\vb{J}_{z}}+\comm{\vb{J}_{x}}{\vb{J}_{z}}\vb{J}_{y}\\
&=i\hbar\qty(-\vb{J}_{x}\vb{J}_{y}-\vb{J}_{y}\vb{J}_{x}+\vb{J}_{y}\vb{J}_{x}+\vb{J}_{x}\vb{J}_{y})\\
&\quad\explain{Using $\comm{\vb{J}_{i}}{\vb{J}_{j}}=\varepsilon_{ijk}(i\hbar\vb{J}_{k}$}
&=0
\end{aligned}
\end{equation}
By similar arguments, $\comm{\vec{\vb{J}}^{2}}{\vb{J}_{x}}=\comm{\vec{\vb{J}}^{2}}{\vb{J}_{y}}=0$. Thus, $\vec{\vb{J}}^{2}$ commutes with all commutes of $\vec{\vb{J}}=(\vb{J}_{x},\vb{J}_{y},\vb{J}_{z})$. This implies $\vec{\vb{J}}^{2}$ and $\vb{J}_{z}$ possess common eigenstates. Note that while $\vec{\vb{J}}^{2}$ commutes 74with all components of $\vec{\vb{J}}$, this does not mean the components commute with each other. One cannot find simultaneous eigenstates between components of $\vec{\vb{J}}$. These eigenstates are specified by
\begin{equation}
\vec{\vb{J}}^{2}\ket{a,b}=a\ket{a,b}\quad\trm{and}\quad\vb{J}_{z}\ket{a,b}=b\ket{a,b}
\end{equation}
Let us denote the following operators $\vb{J}_{\pm}$ as
\begin{equation}
\vb{J}_{\pm}=\vb{J}_{x}\pm i\vb{J}_{y}
\end{equation}
Note that $\vb{J}_{\pm}^{*}=\vb{J}_{\mp}$ are not Hermitian.

\begin{equation}
\begin{aligned}
\comm{\vb{J}_{+}}{\vb{J}_{-}}&=\comm{\vb{J}_{x}+i\vb{J}_{y}}{\vb{J}_{x}-i\vb{J}_{y}}\\
&=-i\comm{\vb{J}_{x}}{\vb{J}_{y}}+i\comm{\vb{J}_{y}}{\vb{J}_{x}}\\
&=2\hbar\vb{J}_{z}\\
\comm{\vb{J}_{z}}{\vb{J}_{\pm}}&=\comm{\vb{J}_{z}}{\vb{J}_{x}\pm i\vb{J}_{y}}\\
&=\comm{\vb{J}_{z}}{\vb{J}_{x}}\pm i\comm{\vb{J}_{z}}{\vb{J}_{y}}\\
&=\pm\hbar\qty(\vb{J}_{x}\pm i\vb{J}_{y})\\
&=\pm\hbar\vb{J}_{\pm}\\
\comm{\vec{\vb{J}}^{2}}{\vb{J}_{\pm}}&=\comm{\vec{\vb{J}}^{2}}{\vb{J}_{x}\pm i\vb{J}_{y}}\\
&=\comm{\vec{\vb{J}}^{2}}{\vb{J}_{x}}\pm i\comm{\vec{\vb{J}}^{2}}{\vb{J}_{y}}\\
&=0
\end{aligned}
\end{equation}
Consider now state $\vb{J}_{\pm}\ket{a,b}$
\begin{equation}
\begin{aligned}
\vb{J}_{z}\vb{J}_{\pm}\ket{a,b}&=\qty(\comm{\vb{J}_{z}}{\vb{J}_{\pm}}+\vb{J}_{\pm}\vb{J}_{z})\ket{a,b}\\
&=(\pm\hbar\vb{J}_{\pm}+\vb{J}_{+}b)\ket{a,b}\\
&=(b\pm\hbar)\vb{J}_{\pm}\ket{a,b}
\end{aligned}
\end{equation}
Thus, $\vb{J}_{\pm}\ket{a,b}$ is still an eigenstate of $\vb{J}_{z}$, but with eigenvalue increased or decreased by $\hbar$. Thus, $\vb{J}_{\pm}$ are called \ul{ladder operators}.
\begin{equation}
\begin{aligned}
\vec{\vb{J}}^{2}\vb{J}_{\pm}\ket{a,b}&=\vb{J}_{\pm}\vec{\vb{J}}^{2}\ket{a,b}\qq{since $\comm{\vec{\vb{J}}^{2}}{\vb{J}_{\pm}}=0$}\\
&=\vb{J}_{\pm}a\ket{a,b}\\
&=a\vb{J}_{\pm}\ket{a,b}
\end{aligned}
\end{equation}
Thus, $\vb{J}_{\pm}\ket{a,b}$ is still an eigenstate of $\vec{\vb{J}}^{2}$ with the eigenvalues unchanged.\\

We denote $\vb{J}_{\pm}\ket{a,b}=c_{\pm}\ket{a,b\pm\hbar}$, where $c_{\pm}$ are coefficients to be determined.\\

We shall now find the upper and lower limits on the eigenvalue $b$. Consider
\begin{equation}
\begin{aligned}
\vec{\vb{J}}^{2}-\vb{J}_{z}^{2}=\vb{J}_{x}^{2}+\vb{J}_{y}^{2}\\
&=\qty[\frac{1}{2}\qty(\vb{J}_{+}+\vb{J}_{-})]^{2}+\qty[\frac{1}{2i}(\vb{J}_{+}-\vb{J}_{-})]^{2}\\
&=\frac{1}{2}(\vb{J}_{+}\vb{J}_{-}+\vb{J}_{-}\vb{J}_{+})\\
&=\frac{1}{2}(\vb{J}_{+}\vb{J}_{+}^{\dagger}+\vb{J}_{+}^{\dagger}\vb{J}_{+})
\end{aligned}
\end{equation}

Note that $\mel{a,b}{\vb{J}_{+}\vb{J}_{+}^{\dagger}}{a,b}$ is the norm of ket $\vb{J}_{+}^{\dagger}\ket{a,b}$ and $\mel{a,b}{\vb{J}_{+}^{\dagger}\vb{J}_{+}}{a,b}$ is the norm of ket $\vb{J}_{+}\ket{a,b}$. Thus
\begin{equation}
\begin{aligned}
\mel{a,b}{(\vec{\vb{J}}^{2}-\vb{J}_{z}^{2})}{a,b}&=\frac{1}{2}\qty[\mel{a,b}{\vb{J}_{+}\vb{J}_{+}^{\dagger}}{a,b}+\mel{a,b}{\vb{J}_{+}^{\dagger}\vb{J}_{+}}{a,b}]\\
&\geq 0
\end{aligned}
\end{equation}
Therefore
\begin{equation}
\begin{aligned}
\mel{a,b}{(\vec{\vb{J}}^{2}-\vb{J}_{z}^{2})}{a,b}&=\mel{a,b}{\vec{\vb{J}}^{2}}{a,b}-\mel{a,b}{\vb{J}_{z}^{2}}{a,b}\\
&=a-b^{2}\\
&\geq0\\
\implies a&\geq b^{2} \implies -\sqrt{a}\leq b \leq \sqrt{a}
\end{aligned}
\end{equation}
Thus, $b$ is bounded. This is merely a consequence of the fact that $\expval{\vec{\vb{J}}^{2}}\geq\expval{\vb{J}_{z}^{2}}$. There must be some maximum eigenvalue $b_{\trm{max}}$ of $\vb{J}_{z}$ such that
\begin{equation}
\begin{aligned}
\vb{J}_{+}\ket{a,b_{\trm{max}}}&=0\\
\therefore \vb{J}_{-}\vb{J}_{+}\ket{a,b_{\trm{max}}}&=\vb{J}_{-}(0)=0
\end{aligned}
\end{equation}
However
\begin{equation}
\begin{aligned}
\vb{J}_{-}\vb{J}_{+}&=(\vb{J}_{x}-i\vb{J}_{y})(\vb{J}_{x}+i\vb{J}_{y})\\
&=\vb{J}_{x}^{2}+\vb{J}_{y}^{2}+i(\vb{J}_{x}\vb{J}_{y}-\vb{J}_{y}\vb{J}_{x})\\
&=\vb{J}_{x}^{2}+\vb{J}_{y}^{2}+i\comm{\vb{J}_{x}}{\vb{J}_{y}}\\
&=\vec{\vb{J}}^{2}-\vb{J}_{z}^{2}-\hbar\vb{J}_{z}\\
\vb{J}_{-}\vb{J}_{+}\ket{a,b_{\trm{max}}}&=(\vec{\vb{J}}^{2}-\vb{J}_{z}^{2}-\hbar\vb{J}_{z})\ket{a,b_{\trm{max}}}\\
&=(a-b_{\trm{max}}^{2}-\hbar b_{\trm{max}})\ket{a,b_{\trm{max}}}\\
&=0\\
\implies a-b_{\trm{max}}^{2}-\hbar b_{\trm{max}}&=0\implies a= b_{\trm{max}}(b_{\trm{max}}+\hbar)
\end{aligned}
\end{equation}
Similarly
\begin{equation}
\begin{aligned}
\vb{J}_{-}\ket{a,b_{\trm{min}}}&=0\\
\vb{J}_{+}\vb{J}_{-}\ket{a,b_{\trm{min}}}&=\vb{J}_{+}(0)=0
\end{aligned}
\end{equation}
However
\begin{equation}
\begin{aligned}
\vb{J}_{+}\vb{J}_{-}&=(\vb{J}_{x}+i\vb{J}_{y})(\vb{J}_{x}-i\vb{J}_{y})\\
&=\vb{J}_{x}^{2}+\vb{J}_{y}^{2}+i(\vb{J}_{y}\vb{J}_{x}-\vb{J}_{x}\vb{J}_{y})\\
&=\vb{J}_{x}^{2}+\vb{J}_{y}^{2}-i\comm{\vb{J}_{x}}{\vb{J}_{y}}\\
&=\vec{\vb{J}}^{2}-\vb{J}_{z}^{2}+\hbar\vb{J}_{z}\\
\vb{J}_{+}\vb{J}_{-}\ket{a,b_{\trm{min}}}&=(\vec{\vb{J}}^{2}-\vb{J}_{z}^{2}+\hbar\vb{J}_{z})\ket{a,b_{\trm{min}}}\\
&=(a-b_{\trm{min}}^{2}+\hbar b_{\trm{min}})\ket{a,b_{\trm{min}}}\\
&=0\\
\implies a-b_{\trm{min}}^{2}+\hbar b_{\trm{max}}&=0\implies a= b_{\trm{min}}(b_{\trm{min}}-\hbar)
\end{aligned}
\end{equation}
Thus
\begin{equation}
\begin{aligned}
b_{\trm{max}}(b_{\trm{max}}+\hbar)&=b_{\trm{min}}(b_{\trm{min}}-\hbar)\\
b_{\trm{min}}&=-b_{\trm{max}}
\end{aligned}
\end{equation}
Since one must be able to reach $\ket{a,b_{\trm{max}}}$ by applying $\vb{J}_{+}$ a finite $n$ amount of times to $\ket{a,b_{\trm{min}}}$
\begin{equation}
\begin{aligned}
\vb{J}_{+}^{n}\ket{a,b_{\trm{min}}}&=(b_{\trm{min}}+n\hbar)\ket{a,b_{\trm{max}}}=b_{\trm{max}}\ket{a,b_{\trm{max}}}\\
b_{\trm{min}}+n\hbar&=b_{\trm{max}}\\
2b_{\trm{max}}&=n\hbar\\
b_{\trm{max}}&=\frac{n\hbar}{2}
\end{aligned}
\end{equation}
Defining $j=\frac{b_{\trm{max}}}{\hbar}$, one finds that $j=\frac{n}{2}$ is an integer or half-integer. The maximum eigenvalue of $\vb{J}_{z}$ is $\hbar j$. The eigenvalue of $\vec{\vb{J}}^{2}$ becomes $a=b_{\trm{max}}(b_{\trm{max}}+\hbar)=\hbar^{2}j(j+1)$.\\

For a classical vector, one expects $a=\hbar^{2}j^{2}$. Note that as $j$ increases, the angular momentum becomes more and more classical.\\

Let $b=m\hbar$, $m\in\{-j,-j+1,\ldots,j-1,j\}$. If $j$ is an integer, then all values of $m$ are integers. If $j$ is a half-integer, then all values of $m$ are half-integers. There are $2j+1$ distinct values for $m$. We replace the eigenequations for $\vec{\vb{J}}^{2}$ and $\vb{J}_{z}$ with
\begin{equation}
\begin{aligned}
\vec{\vb{J}}^{2}\ket{j,m}&=\hbar^{2}j(j+1)\ket{j,m}\\
\vb{J}_{z}\ket{j,m}&=\hbar m\ket{j,m}
\end{aligned}
\end{equation}
where $-j\leq m\leq j$ and $j=\frac{n}{2}$ is the magnitude of the angular momentum (e.g. $j=\frac{1}{2}$ for an electron).\\

The matrix elements of the angular momentum operators are
\begin{equation}
\begin{aligned}
\mel{j',m'}{\vec{\vb{J}}^{2}}{j,m}&=\hbar^{2}j(j+1)\delta_{jj'}\delta_{mm'}\\
\mel{j',m'}{\vb{J}_{z}}{j,m}&=\hbar m\delta_{jj'}\delta_{mm'}
\end{aligned}
\end{equation}
The expectation value for $\vb{J}_{-}\vb{J}_{+}$ is
\begin{equation}
\begin{aligned}
\mel{j,m}{\vb{J}_{-}\vb{J}_{+}}{j,m}&=\mel{j,m}{(\vec{\vb{J}}^{2}-\vb{J}_{z}^{2}-\hbar\vb{J}_{z})}{j,m}\\
&=\hbar^{2}j(j+1)-\hbar^{2}m^{2}-\hbar^{2}m\\
&=\hbar^{2}j(j+1)-\hbar^{2}m(m+1)
\end{aligned}
\end{equation}
However, note that $\vb{J}_{-}\vb{J}_{+}=\vb{J}_{+}^{\dagger}\vb{J}_{+}$. Thus
\begin{equation}
\begin{aligned}
\mel{j,m}{\vb{J}_{-}\vb{J}_{+}}{j,m}&=\mel{j,m}{\vb{J}_{+}^{\dagger}\vb{J}_{+}}{j,m}\\
&=\norm{\vb{J}_{+}\ket{j,m}}^{2}\\
&=\norm{c_{jm}^{+}\ket{j,m}}^{2}\\
&=\abs{c_{jm}^{+}}^{2}\norm{\ket{j,m}}^{2}\\
&=\abs{c_{jm}^{+}}^{2}
\end{aligned}
\end{equation}
Taking $c_{jm}^{+}$ to be real
\begin{equation}
\begin{aligned}
c_{jm}^{+}=\hbar\sqrt{(j-m)(j+m+1)}
\end{aligned}
\end{equation}
Analogously, one can find for $c_{jm}^{-}$ that
\begin{equation}
c_{jm}^{-}=\hbar\sqrt{(j+m)(j-m+1)}
\end{equation}
Thus
\begin{equation}
\begin{aligned}
\vb{J}_{\pm}\ket{j,m}&=\hbar\sqrt{(j\mp m)(j\pm m+1)}\ket{j,m\pm1}\\
\mel{j',m'}{\vb{J}_{\pm}}{j,m}&=\hbar\sqrt{(j\mp m)(j\pm m+1)}\delta_{jj'}\delta_{m'(m\pm1)}
\end{aligned}
\end{equation}
Given the definition of $\vb{J}_{+}$ and $\vb{J}_{-}$
\begin{equation}
\begin{aligned}
\vb{J}_{x}&=\frac{1}{2}(\vb{J}_{+}+\vb{J}_{-})\\
\vb{J}_{y}&=\frac{1}{2i}(\vb{J}_{+}-\vb{J}_{-})
\end{aligned}
\end{equation}
Thus, one can find the matrix elements of $\vb{J}_{x}$ and $\vb{J}_{y}$ in addition to $\vb{J}_{z}$.\\

Returning to the rotation operator $D(R)=e^{-\frac{i}{\hbar}\vec{\vb{J}}\cdot\hat{n}\varphi}$, since $\vec{\vb{J}}^{2}$ commutes with any component of $\vec{\vb{J}}$
\begin{equation}
\comm{\vec{\vb{J}}^{2}}{D(R)}=0
\end{equation}
Therefore
\begin{equation}
\begin{aligned}
\vec{\vb{J}}^{2}D(R)\ket{j,m}&=D(R)\vec{\vb{J}}^{2}\ket{j,m}\\
&=\hbar^{2}j(j+1)D(R)\ket{j,m}
\end{aligned}
\end{equation}
$D(R)\ket{j,m}$ is still an eigenket of $\vec{\vb{J}}^{2}$ with the same eigenvalue $\hbar^{2}j(j+1)$. However, $D(R)$ in general does not commute with $\vb{J}_{z}$ (i.e. $\comm{D(R)}{\vb{J}_{z}}\neq 0$).
\begin{equation}
\begin{aligned}
D(R)\ket{j,m}&=\mathbb{I}D(R)\ket{j,m}\\
&=\sum_{j',m'}\op{j',m'}D(R)\ket{j,m}\\
&=\sum_{m'}\op{j,m'}D(r)\ket{j,m}\qq{by orthogonality}\\
&=\sum_{m'}\ket{j,m'}D_{m'm}^{(j)}(R)
\end{aligned}
\end{equation}
where $D_{m'm}^{(j)}(R)$ is defined to be $\mel{j,m'}{D(R)}{j,m}$.\\

$D^{(j)}(R)$ is a $2j+1$-dimensional irreducible representation of one group of rotations. $\mel{j',m'}{D(R)}{j,m}$ is proportional to $\delta_{j'j}$. This implies that the matrix $\mel{j,m'}{D(R)}{j,m}$ is block diagonal, with each block corresponding to a specific $j$. The size of each block is $2j+1$. $D_{m'm}^{(j)}(R)$ is a probability amplitude to find the system in state $\ket{j,m'}$ after rotation of the original state $\ket{j,m}$.\\

Consider $j=\frac{1}{2}$ case
\begin{equation}
\begin{aligned}
D(R)&=e^{-\frac{i}{\hbar}\vec{\vb{J}}\cdot\hat{n}\varphi}\\
&=e^{-\frac{i}{2}\vec{\sigma}\cdot\hat{n}\varphi}\\
&=\cos\qty(\frac{\varphi}{2})\mathbb{I}-i\vec{\sigma}\cdot\hat{n}\sin\qty(\frac{\varphi}{2})\\
&=\mqty[\cos\qty(\frac{\varphi}{2})-in_{z}\sin\qty(\frac{\varphi}{2}) & (-in_{x}-n_{y})\sin\qty(\frac{\varphi}{2})\\ (in_{x}-n_{y})\sin\qty(\frac{\varphi}{2}) & \cos\qty(\frac{\varphi}{2})+in_{z}\sin\qty(\frac{\varphi}{2})]
\end{aligned}
\end{equation}

\subsection{Orbital and Spin Angular Momentum}
In classical mechanics, angular momentum is what is know as \ul{orbital angular momentum}, defined as
\begin{equation}
\vec{\vb{L}}=\vec{\vb{r}}\times\vec{\vb{p}}
\end{equation}
where $\vec{\vb{r}}$ and $\vec{\vb{p}}$ are both operators.\\

In quantum mechanics, the total angular momentum is the sum of the orbital angular momentum and the \ul{spin}
\begin{equation}
\vec{\vb{J}}=\vec{\vb{L}}+\vec{\vb{S}}
\end{equation}
Consider the commutation relations for $\vec{\vb{L}}$
\begin{equation}
\begin{aligned}
\comm{\vb{L}_{x}}{\vb{L}_{y}}&=\comm{\vb{y}\vb{p}_{z}-\vb{z}\vb{p}_{y}}{\vb{z}\vb{p}_{x}-\vb{x}\vb{p}_{z}}\\
&=\comm{\vb{y}\vb{p}_{z}}{\vb{z}\vb{p}_{x}}-\cancelto{0}{\comm{\vb{y}\vb{p}_{z}}{\vb{x}\vb{p}_{z}}}-\cancelto{0}{\comm{\vb{z}\vb{p}_{y}}{\vb{z}\vb{p}_{x}}}+\comm{\vb{z}\vb{p}_{y}}{\vb{x}\vb{p}_{z}}\\
&\quad\explain{Since $\vb{x}\neq\vb{y}$, $\vb{p}_{z}=\vb{p}_{z}$ and $\vb{z}=\vb{z}$, $\vb{p}_{y}\neq\vb{p}_{x}$}\\
&\quad\explain{$\comm{\vb{p}_{i}}{\vb{p}_{j}}=\comm{\vb{x}_{i}}{\vb{x}_{j}}=0$, $\comm{\vb{x}_{i}}{\vb{p}_{j}}=\delta_{ij}$}\\
&=\ldots\\
&=\vb{y}\vb{p}_{x}\underbrace{\comm{\vb{p}_{z}}{\vb{z}}}_{-i\hbar}+\vb{p}_{y}\vb{x}\underbrace{\comm{\vb{z}}{\vb{p}_{z}}}_{i\hbar}\\
&=i\hbar(\vb{x}\vb{p}_{y}-\vb{y}\vb{p}_{x})\\
&=i\hbar\vb{L}_{z}
\end{aligned}
\end{equation}

Similar relations can be found for other combinations, yielding the canonical commutation relation
\begin{equation}
\comm{\vb{L}_{i}}{\vb{L}_{j}}=i\hbar\varepsilon_{ijk}\vb{L}_{k}
\end{equation}
Consider spin-less infinitesimal rotations $(\vec{\vb{S}}=\vec{0})$. In this case, $\vec{\vb{J}}=\vec{\vb{L}}$.
\begin{equation}
D_{z}(\delta\varphi)=e^{-\frac{i}{\hbar}\vb{L}_{z}\delta\varphi}=\mathbb{I}-\frac{i}{\hbar}\vb{L}_{z}\delta\varphi=\mathbb{I}-\frac{i}{\hbar}\delta\varphi(\vb{x}\vb{p}_{y}-\vb{y}\vb{p}_{x})
\end{equation}
Acting on a position eigenket $\ket{\vec{x}\,'}$
\begin{equation}
D_{z}(\delta\varphi)\ket{\vec{x}\,'}=\qty(\mathbb{I}-\frac{i}{\hbar}\delta\varphi\vb{x}\vb{p}_{y}+\frac{i}{\hbar}\delta\varphi\vb{y}\vb{p}_{x})\ket{x',y',z'}
\end{equation}

Recall that
\begin{equation}
\begin{aligned}
\vb{T}(\dd{\vec{x}}\,')&=\mathbb{I}-\frac{i}{\hbar}\vec{p}\cdot\dd{\vec{x}}\,'\\
\vb{T}(\dd{\vec{x}}\,')\ket{\vec{x}\,'}&=\ket{\vec{x}\,'+\dd{\vec{x}}\,'}
\end{aligned}
\end{equation}
Thus
\begin{equation}
D_{z}(\delta\varphi)\ket{\vec{x}\,'}=\ket{x'-\delta\varphi\,y',y'+\delta\varphi\,x',z'}
\end{equation}
Another way to see this is
\begin{equation}
\begin{aligned}
D_{z}(\delta\varphi)\ket{\vec{x}\,'}&=\qty(\mathbb{I}-\frac{i}{\hbar}\delta\varphi\,x' \vb{p}_{y}+\frac{i}{\hbar}\delta\varphi\, y'\vb{p}_{x})\ket{x',y',z'}\\
&=\ket{x'-\delta\varphi\,y',y'+\delta\varphi\,x',z'}
\end{aligned}
\end{equation}
which is exactly the effect of rotating a vector $\vec{x}\,'$ by the infinitesimal rotation matrix $R_{z}(\delta\varphi)$.
\begin{equation}
\begin{aligned}
R_{z}(\delta\varphi)\vec{x}\,'&=\mqty[\cos\delta\varphi &-\sin\delta\varphi & 0\\ \sin\delta\varphi & \cos\delta\varphi & 0 \\ 0 & 0 & 1]\mqty[x' \\ y' \\ z']\\
&\approx \mqty[1 & -\delta\varphi & 0 \\ \delta\varphi & 1 & 0 \\ 0 & 0 & 1]\mqty[x' \\ y' \\ z']\\\\
&=\mqty[x'-\delta\varphi\,y'\\ y' + \delta\varphi\,x' \\ z']
\end{aligned}
\end{equation}
Consider a state $\ket{\psi}$ of a paritcle in the coordinate representation with wavefunction
\begin{equation}
\psi(\vec{x}\,')=\ip{\vec{x}\,'}{\psi}=\ip{x',y',z'}{\psi}
\end{equation}
After rotation $\delta\varphi$ around the $z$-axis
\begin{equation}
\begin{aligned}
\mel{x',y',z'}{D_{z}(\delta\varphi)}{\psi}&=\mel{x',y',z'}{\qty(\mathbb{I}-\frac{i}{\hbar}\vb{L}_{z}\delta\varphi)}{\psi}\\
&=\mel{\psi}{\qty(\mathbb{I}+\frac{i}{\hbar}\vb{L}_{z}\delta\varphi)}{x',y',z'}^{*}\\
&=\ip{\psi}{x'+\delta\varphi\,y',y'-\delta\varphi\,x',z'}^{*}\\
&=\ip{x'+\delta\varphi\,y',y'-\delta\varphi\,x',z'}{\psi}
\end{aligned}
\end{equation}
This may be more easily represented in spherical coordinates
\begin{equation}
\begin{aligned}
\vec{x}\,'&=x'\hat{x}+y'\hat{y}+z'\hat{z}=r\hat{r}\\
x'&=r\sin\theta\cos\varphi\\
y'&=r\sin\theta\sin\varphi\\
z'&=r\cos\theta\\
r&=\sqrt{(x')^{2}+(y')^{2}+(z')^{2}}
\end{aligned}
\end{equation}
Notice to first order, for a rotation $\delta\varphi$ about the $z$-axis
\begin{equation}
\begin{aligned}
x'&\rightarrow r\sin\theta\cos(\varphi-\delta\varphi)\approx r\sin\theta\cos\varphi+r\sin\theta\sin\phi \,\delta\varphi = x'+\delta\varphi\,y'\\
y'&\rightarrow r\sin\theta\sin(\varphi-\delta\varphi)\approx r\sin\theta\sin\varphi-r\sin\theta\cos\varphi\,\delta\varphi=y'-\delta\varphi\, x'
z'&\rightarrow z
\end{aligned}
\end{equation}
The effect of rotation about the $z$-axis is $\varphi\rightarrow\varphi-\delta\varphi$
\begin{equation}
\begin{aligned}
\mel{r,\theta,\varphi}{\qty(\mathbb{I}-\frac{i}{\hbar}\vb{L}_{z}\delta\varphi)}{\psi}&=\ip{r,\theta,\varphi-\delta\varphi}{\psi}\\
&\approx\ip{r,\theta,\varphi}{\psi}-\delta\varphi\,\pdv{\varphi}\ip{r,\theta,\varphi}{\psi}\\
&=\ip{r,\theta,\varphi}{\psi}-\frac{i}{\hbar}\delta\varphi\,\mel{r,\theta,\varphi}{\vb{L}_{z}}{\psi}
\end{aligned}
\end{equation}
Thus, the coordinate representation of $\vb{L}_{z}$ is
\begin{equation}
\vb{L}_{z}=-i\hbar\pdv{\varphi}
\end{equation}
More generally, $\vec{\vb{L}}=\vec{\vb{r}}\times\vec{\vb{p}}=-i\hbar\vec{\vb{r}}\times\vec{\nabla}$. In spherical coordinates
\begin{equation}
\begin{aligned}
\hat{r}&=\hat{x}\sin\theta\cos\varphi+\hat{y}\sin\theta\sin\varphi+\hat{z}\cos\theta\\
\hat{\varphi}&=-\hat{x}\sin\varphi+\hat{y}\cos\varphi\\
\hat{\theta}&=\hat{x}\cos\varphi\cos\theta+\hat{y}\sin\varphi\cos\theta-\hat{z}\sin\theta
\end{aligned}
\end{equation}
This can be used to determine
\begin{equation}
\begin{aligned}
\vb{L}_{x}&=\vb{y}\vb{p}_{z}-\vb{z}\vb{p}_{y}=-i\hbar\qty(\vb{y}\pdv{z}-\vb{z}\pdv{y})\\
\vb{L}_{y}&=\vb{z}\vb{p}_{x}-\vb{x}\vb{p}_{z}=-i\hbar\qty(\vb{z}\pdv{x}-\vb{x}\pdv{z})
\end{aligned}
\end{equation}
Thus
\begin{equation}
\begin{aligned}
\vec{\nabla}&=\hat{x}\pdv{x}+\hat{y}\pdv{y}+\hat{z}\pdv{z}\\
&=\hat{r}\pdv{r}+\hat{\varphi}\frac{1}{r\sin\theta}\pdv{\varphi}+\hat{\theta}\frac{1}{r}\pdv{\theta}
\end{aligned}
\end{equation}
Note that $\hat{\theta}=\hat{\varphi}\times\hat{r}$. Thus
\begin{equation}
\begin{aligned}
\pdv{z}&=\cos\theta\pdv{r}-\frac{1}{r\sin\theta}\pdv{\theta}\\
\pdv{y}&=\sin\theta\sin\varphi\pdv{r}+\frac{\cos\varphi}{r\sin\theta}\pdv{\varphi}+\frac{\cos\theta\sin\varphi}{r}\pdv{\theta}\\
\pdv{x}&=\sin\theta\cos\varphi\pdv{r}-\frac{\sin\varphi}{r\sin\theta}\pdv{\varphi}+\frac{\cos\theta\cos\varphi}{r}\pdv{\theta}
\end{aligned}
\end{equation}
and
\begin{equation}
\begin{aligned}
\vb{L}_{x}&=-i\hbar\qty(-\sin\varphi\pdv{\theta}-\cot\theta\cos\varphi\pdv{\varphi})\\
\vb{L}_{y}&=-i\hbar\qty(\cos\varphi\pdv{\theta}-\cot\theta\sin\varphi\pdv{\varphi})\\
\vb{L}_{z}&=-i\hbar\pdv{\varphi}
\end{aligned}
\end{equation}
Note that all components of $\vec{\vb{L}}$ only contain derivatives with respect to $\theta$ and $\varphi$ (and not $r$) since rotations do not change the length of $r$.
\begin{equation}
\begin{aligned}
\vb{L}_{\pm}&=\vb{L}_{x}\pm i\vb{L}_{y}=-i\hbar e^{\pm i\varphi}\qty(\pm i\pdv{\theta}-\cot\theta\pdv{\varphi})\\
\vec{\vb{L}}^{2}&=\vb{L}_{x}^{2}+\vb{L}_{y}^{2}+\vb{L}_{z}^{2}\\
&=\vb{L}_{z}^{2}+\frac{1}{2}\qty(\vb{L}_{+}\vb{L}_{-}+\vb{L}_{-}\vb{L}_{+})\\
&=\ldots\\
&=-\hbar^{2}\qty[\frac{1}{\sin^{2}\theta}\pdv[2]{\varphi}+\frac{1}{\sin\theta}\pdv{\theta}\qty(\sin\theta\pdv{\theta})]
\end{aligned}
\end{equation}
Moreover, note that $\comm{\vec{\vb{L}}^{2}}{\vb{L}_{z}}=0$, since $\vec{\vb{L}}^{2}$ is a function of $\theta$ only and does not depend on $\varphi$.\\

Let us find common eigenstates for $\vec{\vb{L}}^{2}$ and $\vb{L}_{z}$. Using a similar process as before
\begin{equation}
\begin{aligned}
\vec{\vb{L}}^{2}\ket{l,m}&=\hbar^{2}l(l+1)\ket{l,m}\\
\vb{L}_{z}\ket{l,m}&=\hbar m\ket{l,m}
\end{aligned}
\end{equation}
where $m\in\{-l,-l+1,\ldots,l-1,l\}$ and can be $2l+1$ distinct values.\\

Consider the eigenkets $\ket{l,m}$ in the position representation. For spherical coordinates, they should only depend on the angular variables $\theta$ and $\varphi$, given there are no derivatives in $r$.\\

Thus, we define the follow wavefunctions
\begin{equation}
\ip{\theta,\varphi}{l,m}=Y_{l}^{m}(\theta,\varphi)
\end{equation}
to be the \ul{spherical harmonics}.\\

$\abs{Y_{l}^{m}(\theta,\varphi)}^{2}$ is the probability of finding particle in state $\ket{l,m}$ at $\ket{\theta,\phi}$.
\begin{equation}
\begin{aligned}
\mel{\theta,\varphi}{\vb{L}_{z}}{l,m}&=\hbar m\ip{\theta,\varphi}{l,m}=\hbar mY_{l}^{m}(\theta,\varphi0\\
\mel{\theta,\varphi}{\vb{L}_{z}}{l,m}&=\qty(-i\hbar\pdv{\varphi})\ip{\theta,\varphi}{l,m}=\qty(-i\hbar\pdv{\varphi})Y_{l}^{m}(\theta,\varphi)\\
&\quad\explain{By resolution of identity}\\
\implies \qty(-i\hbar\pdv{\varphi})Y_{l}^{m}(\theta,\varphi)&=\hbar mY_{l}^{m}(\theta,\varphi)
\end{aligned}
\end{equation}
This differential equation implies that the $\varphi$-dependence of $Y_{l}^{m}(\theta,\varphi)$ goes like $Y_{l}^{m}(\theta,\varphi)\sim e^{im\varphi}$.

\begin{equation}
\begin{aligned}
\vec{\vb{L}}^{2}\ket{l,m}&=\hbar^{2}l(l+1)\ket{l,m}\\
-\hbar^{2}\qty[\frac{1}{\sin^{2}\theta}\pdv[2]{\varphi}+\frac{1}{\sin\theta}\pdv{\theta}\qty(\sin\theta\pdv{\theta})]Y_{l}^{m}(\theta,\varphi)&=\hbar^{2}l(l+1)Y_{l}^{m}(\theta,\varphi)
\end{aligned}
\end{equation}
Note that $\qty(-i\hbar\pdv{\varphi})Y_{l}^{m}(\theta,\varphi)=\hbar mY_{l}^{m}(\theta,\varphi)$.
\begin{equation}
\therefore\qty[\frac{1}{\sin\theta}\pdv{\theta}\qty(\sin\theta\pdv{\theta})-\frac{m^{2}}{\sin^{2}\theta}+l(l+1)]Y_{l}^{m}(\theta,\varphi)=0
\end{equation}
This yields the spherical harmonics solution for $Y_{l}^{m}(\theta,\varphi)$
\begin{equation}
Y_{l}^{m}(\theta,\varphi)=(-1)^{m}\sqrt{\frac{(2l+1)(l-m)!}{4\pi(l+m)!}}P_{l}^{m}(\cos\theta)e^{im\varphi}\qq{for $m>0$}
\end{equation}
where
\begin{equation}
P_{l}^{m}(\cos\theta)=\frac{(-1)^{m+l}}{2^{l}l!}\frac{(l+\abs{m})!}{(l-\abs{m})!}\frac{1}{\sin^{\abs{m}}\theta}\qty(\dv{(\cos\theta)})^{l-\abs{m}}\sin^{2l}\theta
\end{equation}
are the \ul{associated Legendre polynomials}.\\

This has the property of
\begin{equation}
Y_{l}^{-m}(\theta,\varphi)=(-1)^{l}Y_{l}^{m}{}^{\dagger}(\theta,\varphi)
\end{equation}
Under parity (spatial inversion)
\begin{equation}
\begin{aligned}
\theta&\rightarrow\pi-\theta\\
\varphi&\rightarrow\varphi+\pi\\
e^{im\varphi}&\rightarrow e^{im\varphi}e^{im\pi}=(-1)^{m}e^{im\varphi}\\
\cos\theta&\rightarrow-\cos\theta
\end{aligned}
\end{equation}
Thus
\begin{equation}
P_{l}^{m}(-\cos\theta)=(-1)^{l-m}P_{l}^{m}(\cos\theta)
\end{equation}
Under parity
\begin{equation}
\begin{aligned}
Y_{l}^{m}(\pi-\theta,\varphi+\pi)&=(-1)^{m}(-1)^{l-m}Y_{l}^{m}(\theta,\varphi)\\
&=(-1)^{l}Y_{l}^{m}(\theta,\varphi)
\end{aligned}
\end{equation}
For even $l$, $Y_{l}^{m}(\theta,\varphi)$ doesn't change under inversion. For odd $l$, $Y_{l}^{m}(\theta,\varphi)$ changes sign under inversion.\\

Remember for total angular momentum $\vec{\vb{J}}$, $j$ was either an integer or half-integer. This is not true for $l$. Take $Y_{l}^{m}(\theta,\varphi)$ and consider a rotation of $2\pi$ for $\varphi$. By rotational symmetry
\begin{equation}
\begin{aligned}
Y_{l}^{m}(\theta,\varphi)&=Y_{l}^{m}(\theta,\varphi+2\pi)\\
Y_{l}^{m}(\theta,\varphi)&=e^{2\pi im}Y_{l}^{m}(\theta,\varphi)\\
1&=e^{2\pi im}
\end{aligned}
\end{equation}
If $l$ is a half-integer, then all $m$ are half-integer and $e^{2\pi im}=-1$, which leads to a contradiction. Thus, $l$ and consequently $m$ must be integers. The half-integer component of $\vec{\vb{J}}$ comes entirely from $\vec{\vb{S}}$. $\vec{\vb{L}}$ has a classical analogue, while $\vec{\vb{S}}$ has no classical analogue.

\newpage
\section{Symmetries in Quantum Mechanics}
Transformations such as translations, rotation, and time-evolution are associated with unitary operators. We shall call these \ul{symmetry operators}. Suppose $\vb{S}$ is a symmetry operator. For an infinitesimal transformation
\begin{equation}
\vb{S}=\mathbb{I}-\frac{i}{\hbar}\vb{G}\epsilon
\end{equation}
where $\vb{G}$ is a Hermitian operator (as a consequence of $\vb{S}$ being unitary) and $\epsilon\ll 1$. $\vb{G}$ is the \ul{generator} of transformation $\vb{S}$. For example
\begin{equation}
D(\hat{n},\delta\varphi)=\mathbb{I}-\frac{i}{\hbar}\vec{\vb{J}}\cdot\hat{n}\delta\varphi
\end{equation}

Suppose the Hamiltonian $\vb{H}$ is invariant under $\vb{S}$
\begin{equation}
\vb{S}^{\dagger}\vb{H}\vb{S}=\vb{H}
\end{equation}
(e.g. a particle in a central potential $\vb{H}=\frac{(\vec{\vb{p}})^{2}}{2m}+V(\vb{r})$, is invariant under rotation $D^{\dagger}\vb{H}D=\vb{H}$. Then
\begin{equation}
\begin{aligned}
\vb{S}^{\dagger}\vb{H}\vb{S}&=\qty(\mathbb{I}+\frac{i}{\hbar}\epsilon\vb{G})\vb{H}\qty(\mathbb{I}-\frac{i}{\hbar}\epsilon\vb{G})\\
&=\vb{H}+\frac{i\epsilon}{\hbar}\vb{G}\vb{H}-\frac{i\epsilon}{\hbar}\vb{H}\vb{G}+\cancelto{0}{\order{\epsilon^{2}}}\\
&=\vb{H}+\frac{i\epsilon}{\hbar}\comm{\vb{G}}{\vb{H}}
\end{aligned}
\end{equation}
Since $\vb{S}^{\dagger}\vb{H}\vb{S}=\vb{H}$ (with $\vb{S}$ being a symmetry), then $\comm{\vb{G}}{\vb{H}}=0$. Note in the central potential example, $\comm{\vec{\vb{J}}}{\vb{H}}=0$. By the Heisenberg equation of motion
\begin{equation}
\dv{\vb{G}}{t}=\frac{i}{\hbar}\comm{\vb{H}}{\vb{G}}=0
\end{equation}
Thus, when $\vb{H}$ is invariant with respect to a symmetry transformation $\vb{S}$, its generator $\vb{G}$ is a conserved quantity.\\

Since $\vb{S}$ is unitary
\begin{equation}
\begin{aligned}
\vb{S}^{\dagger}\vb{H}\vb{S}&=\vb{H}\qq{if $\vb{H}$ is invariant under transformation $\vb{S}$}\\
\vb{S}^{-1}\vb{H}\vb{S}&=\vb{H}\\
\vb{H}\vb{S}&=\vb{S}\vb{H}\\
\comm{\vb{H}}{\vb{S}}&=0
\end{aligned}
\end{equation}
Consider energy eigenkets $\ket{n}$ of $\vb{H}$ with eigenvalues $E_{n}$
\begin{equation}
\begin{aligned}
\vb{H}\ket{n}&=E_{n}\ket{n}\\
\vb{H}\vb{S}\ket{n}&=\vb{S}\vb{H}\ket{n}=\vb{S}E_{n}\ket{n}=E_{n}\vb{S}\ket{n}
\end{aligned}
\end{equation}
Thus, $\vb{S}\ket{n}$ is also an eigenket of $\vb{H}$ with eigenvalue $E_{n}$. Thus, $\ket{n}$ and $\vb{S}\ket{n}$ are degenerate eigenkets of $\vb{H}$. Symmetries are associated with degeneracies. For ``energy level repulsions", there is an absence of symmetries. There are usually no degeneracies.\\

In the case of the central potential system, the Hamiltonian is rotationally invariant.
\begin{equation}
D^{\dagger}(R)\vb{H}D(R)=\vb{H}\implies\comm{\vb{H}}{D(R)}=0
\end{equation}
This also implies
\begin{equation}
\comm{\vb{H}}{\vec{\vb{J}}^{2}}=0,\quad\comm{\vb{H}}{\vec{\vb{J}}}=0
\end{equation}
Thus, one can find simultaneous eigenkets $\ket{n,j,m}$ of $\vb{H}$, $\vec{\vb{J}}^{2}$, and $\vb{J}_{z}$.
\begin{equation}
\vb{H}\ket{n,j,m}=E_{n}\ket{n,j,m}
\end{equation}
Since $\comm{\vb{H}}{D(R)}=0$, $D(R)\ket{n,j,m}$ is also an eigenstate of $\vb{H}$ with the same energy. Thus, one has
\begin{equation}
\begin{aligned}
D(R)\ket{n,j,m}&=\sum_{m'}\op{n,j,m'}D(R)\ket{n,j,m}\\
&=\sum_{m'}\ket{n,j,m'}D^{(j)}_{m'm}(R)
\end{aligned}
\end{equation}
If all states of this form, for an arbitrary rotation $D(R)$ are eigenstates with the same energy, it must be true that all $\ket{n,j,m}$ with different $m\in\{-j,\ldots,j\}$ have the same energy. Thus, in the presence of rotational symmetry, all eigenstates are $2j+1$-fold degenerate. These rotations are an example of continuous symmetries. The rotation angle can be changed continuously.

\subsection{Discrete Symmetries}
\subsubsection{Parity Symmetry}
Parity changes a right-handed coordinate system to a left-handed coordinate system.
\begin{center}
\begin{tikzpicture}
	\draw[->] (0,0,0) -- (2,0,0) node[right] {$\hat{x}$};
	\draw[->] (0,0,0) -- (0,2,0) node[above] {$\hat{y}$};
	\draw[->] (0,0,0) -- (0,0,2) node[below left] {$\hat{z}$};
	\node at (3,1,0) {$\hat{z}=\hat{x}\times\hat{y}$};

	\draw[->] (8,0,0) -- (6,0,0) node[left] {$\hat{x}$};
	\draw[->] (8,0,0) -- (8,-2,0) node[below] {$\hat{y}$};
	\draw[->] (8,0,0) -- (8,0,-2) node[above right] {$\hat{z}$};
	\node at (11,1,0) {$\hat{z}=-\hat{x}\times\hat{y}$};
\end{tikzpicture}
\end{center}
In essence, $\vec{r}\rightarrow-\vec{r}$. Inversion change handedness, while rotation preserve handedness.\\

As usual, consider transformations of kets and not the coordinate system, which remains fixed. Under the action of the parity operator, $\ket{\psi}\rightarrow\Pi\ket{\psi}$.\\

By definition, the expectation value of the position operator $\vec{\vb{x}}$ changes sign under the parity operator $\Pi$. Thus
\begin{equation}
\begin{aligned}
\mel{\psi}{\Pi^{\dagger}\vec{\vb{x}}\Pi}{\psi}&=-\mel{\psi}{\vec{\vb{x}}}{\psi}\\
\pi^{\dagger}\vec{\vb{x}}\Pi&=-\vec{\vb{x}}\\
\vec{\vb{x}}\Pi&=-\Pi\vec{\vb{x}}\qq{since $\Pi$ is unitary from $\mel{\psi}{\Pi^{\dagger}\Pi}{\psi}=\ip{\psi}=1$}
\end{aligned}
\end{equation}
This can be written as $\acomm{\vec{\vb{x}}}{\Pi}=0$, where $\acomm{\vec{\vb{x}}}{\Pi}=\vec{\vb{x}}\Pi+\Pi\vec{\vb{x}}$ is the \ul{anti-commutator}.\\

For $\Pi$ acting on the position eigenket $\ket{\vec{x}\,'}$
\begin{equation}
\vec{\vb{x}}\Pi\ket{\vec{x}\,'}=-\Pi\vec{\vb{x}}\ket{\vec{x}\,'}=-\vec{x}\,'\Pi\ket{\vec{x}\,'}
\end{equation}
Thus, $\Pi\ket{\vec{x}\,'}$ is also an eigenket of $\vec{\vb{x}}$ with eigenvalue $-\vec{x}\,'$. Thus
\begin{equation}
\Pi\ket{\vec{x}\,'}=e^{i\delta}\ket{-\vec{x}\,'}\qq{where $e^{i\delta}$ is an arbitrary phase factor}
\end{equation}
We shall choose $e^{i\delta}=1$ by convention.\\

Additionally
\begin{equation}
\begin{aligned}
\Pi^{2}\ket{\vec{x}\,'}&=\Pi\ket{\vec{x}\,'}=\ket{\vec{x}\,'}\\
\therefore\Pi^{2}&=\mathbb{I}\Longleftrightarrow\Pi^{-1}=\Pi
\end{aligned}
\end{equation}
Since $\Pi^{-1}=\Pi$ and $\Pi^{-1}=\Pi^{\dagger}$, $\Pi=\Pi^{\dagger}$. Thus, $\Pi$ is Hermitian and unitary. Since $\Pi^{2}=\mathbb{I}$, the eigenvalues of $\Pi$ must be $\pm1$.\\

Consider the effects of parity on translation $T(\dd{\vec{x}}\,')=\mathbb{I}-\frac{i}{\hbar}\vec{\vb{p}}\cdot\dd{\vec{x}}\,'$.
\begin{center}
\begin{tikzpicture}
	\draw[<->] (-4,0) -- (4,0) node[right] {$x$};
	\draw[<->] (0,-3) -- (0,3) node[above] {$y$};
	\draw[<->] (-4,-3) -- (4,3);
	\draw[dashed] (-4,-2) -- (4,2);
	\draw[->] (-4,-2) -- node[left] {$-\dd{\vec{x}}\,'$} (-4,-3);
	\draw[->] (4,2) -- node[right] {$\dd{\vec{x}}\,'$} (4,3);
\end{tikzpicture}
\end{center}
Thus
\begin{equation}
\begin{aligned}
\Pi T(\dd{\vec{x}}\,')&=T(-\dd{\vec{x}}\,')\Pi\\
\implies \Pi T(\dd{\vec{x}}\,')\Pi^{\dagger}&=T(-\dd{\vec{x}}\,')\\
\Pi\qty(\mathbb{I}-\frac{i}{\hbar}\vec{\vb{p}}\cdot\dd{\vec{x}}\,')\Pi^{\dagger}&=\mathbb{I}+\frac{i}{\hbar}\vec{\vb{p}}\cdot\dd{\vec{x}}\,'\\
\mathbb{I}-\frac{i}{\hbar}\Pi\vec{\vb{p}}\cdot\dd{\vec{x}}\,'\Pi^{\dagger}&=\mathbb{I}+\frac{i}{\hbar}\vec{\vb{p}}\cdot\dd{\vec{x}}\,'\\
\Pi\vec{\vb{p}}\cdot\dd{\vec{x}}\,'\Pi^{\dagger}&=-\vec{\vb{p}}\cdot\dd{\vec{x}}\,'\\
\Pi\vec{\vb{p}}\Pi^{\dagger}&=-\vec{\vb{p}}\qq{since $\dd{\vec{x}}\,'$ isn't an operator}
\end{aligned}
\end{equation}
Thus, $\vec{\vb{p}}$ changes sign under parity. Thus, position operator $\vec{\vb{x}}$ and momentum operator $\vec{\vb{p}}$ has odd symmetry under parity. Thus, one can see that $\vec{\vb{L}}$ is even under parity.
\begin{equation}
\vec{\vb{L}}=\vec{\vb{x}}\times\vec{\vb{p}}\rightarrow\Pi^{\dagger}\vec{\vb{L}}\Pi=\vec{\vb{L}}
\end{equation}
More generally, connecting with rotations, parity can be represented as a diagonal matrix.
\begin{equation}
R^{\pi}=\mqty[-1 & 0 & 0\\ 0 & -1 & 0\\ 0 & 0 & -1]
\end{equation}
This clearly commutes with any rotation matrix $R$. Thus
\begin{equation}
\comm{R^{\pi}}{R}=0\implies\comm{\Pi}{D(R)}=0
\end{equation}
For $D(R)=\mathbb{I}-\frac{i}{\hbar}\vec{\vb{J}}\cdot\hat{n}\dd{\varphi}$
\begin{equation}
\begin{aligned}
\Pi D(R)&=D(R)\Pi\\
\Pi^{\dagger}D(R)\Pi&=D(R)\\
\Pi^{\dagger}\qty(\mathbb{I}-\frac{i}{\hbar}\vec{\vb{J}}\cdot\hat{n}\dd{\varphi})\Pi&=\mathbb{I}-\frac{i}{\hbar}\vec{\vb{J}}\cdot\hat{n}\dd{\varphi}\\
\mathbb{I}-\frac{i}{\hbar}\Pi^{\dagger}\vec{\vb{J}}\cdot\hat{n}\Pi\dd{\varphi}&=\mathbb{I}-\frac{i}{\hbar}\vec{\vb{J}}\cdot\hat{n}\dd{\varphi}\\
\Pi^{\dagger}\vec{\vb{J}}\cdot\hat{n}\Pi&=\vec{\vb{J}}\cdot\hat{n}\\
\implies\Pi^{\dagger}\vec{\vb{J}}\Pi&=\vec{\vb{J}}
\end{aligned}
\end{equation}
Thus, $\vec{\vb{J}}$ has been proven to be even under parity. This also implies that $\Pi^{\dagger}\vec{\vb{S}}\Pi=\vec{\vb{S}}$. Thus, spin $\vec{\vb{S}}$ is also even under parity.\\

Vectors which are odd under parity are called \ul{polar vectors}. Those that are even under parity are called \ul{axial vectors} (or \ul{pseudo-vectors}). Scalar operators that change sign under parity are called \ul{pseudo-scalars}




\end{document}